%%%%%%%%%%%%%%%%%%%%%%%%%%%%%%%%%%%
%\documentclass[prd,aps,twocolumn,nofootinbib,showpacs,superscriptaddress]{revtex4-1}
\documentclass[prd,twocolumn,nofootinbib]{revtex4-1}
%\documentclass[prd,aps,nofootinbib,showpacs]{revtex4-1}
\usepackage{amsfonts}
\usepackage{amsmath}
\usepackage{amssymb}
\usepackage{bm}
\usepackage{dcolumn}
\usepackage[dvips]{graphicx}
\usepackage{graphics}
%\usepackage[latin1]{inputenc}
\usepackage{latexsym}
\usepackage{rotating}
\usepackage[colorlinks=true]{hyperref}
\usepackage{xspace} % Sensible space treatment at end of simple macros
\usepackage[usenames]{color}
\usepackage{mathrsfs}
\usepackage{multirow}
\usepackage{pifont}
\usepackage{enumitem}
\usepackage{color}
%\usepackage{ulem}
\usepackage{appendix}
\usepackage[utf8]{inputenc}
%\usepackage[toc,page]{appendix}
\usepackage{comment}
%% Try to control orphans, widows, and extra whitespace
\widowpenalty=1000
\clubpenalty=1000
\raggedbottom

\definecolor {darkgreen}{rgb}{0.2,0.7,0.2}
\definecolor{purple}{rgb}{0.5,0,0.5}

\newcommand\be{\begin{equation}}
\newcommand\ba{\begin{eqnarray}}
\newcommand\ee{\end{equation}}
\newcommand\ea{\end{eqnarray}}
\newcommand\bw{\begin{widetext}}
\newcommand\ew{\end{widetext}}

\newcommand{\EDGB}{{\mbox{\tiny EdGB}}}
\newcommand{\BD}{{\mbox{\tiny BD}}}
\newcommand{\NC}{{\mbox{\tiny NC}}}
\newcommand{\PPE}{{\mbox{\tiny ppE}}}
\newcommand{\KG}{{\mbox{\tiny kh}}}
\newcommand{\EA}{{\mbox{\tiny EA}}}
\newcommand{\ST}{{\mbox{\tiny ST}}}
\newcommand{\NS}{{\mbox{\tiny NS}}}
\newcommand{\DCS}{{\mbox{\tiny dCS}}}
\newcommand{\GR}{{\mbox{\tiny GR}}}
\newcommand{\GW}{{\mbox{\tiny GW}}}
\newcommand{\ky}[1]{\textcolor{blue}{\it{\textbf{ky: #1}}} }
\newcommand{\st}[1]{\textcolor{cyan}{\it{\textbf{st: #1}}} }


\begin{document}
\title{Parameterized Post-Einsteinian Gravitational Waveforms in \\ Various Modified Theories of Gravity}

\author{Sharaban Tahura}
\affiliation{Department of Physics, University of Virginia, Charlottesville, Virginia 22904, USA.}

\author{Kent Yagi}
\affiliation{Department of Physics, University of Virginia, Charlottesville, Virginia 22904, USA.}

\begin{abstract} 

Something goes here 

\end{abstract}

\date{\today}




\maketitle


%%%%%%%%%%%%%%%%%%%%%%%%%%
\section{Introduction}
\ky{Let's not forget to mention somewhere in the paper that we only focus on the quadrupolar tensor modes and there could be additional polarization modes such as scalar breathing mode, though such modes typically enter at higher PN order.}

\ky{Also, we should mention that we focus on modifications in the generation of GWs and PPE parameters for modifications in the propagation of GWs can be found in~\cite{Mirshekari:2011yq,Yunes:2016jcc,Nishizawa:2017nef}.}

%%%%%%%%%%%%%%%%%%%%%%%%%%
\section{ppE Waveform}\label{section:ppE}
We begin by reviewing the ppE formalism. The original formalism (that we explain in detail in App.~\ref{appendix}) was developed by considering non-GR corrections to the binding energy $E$ and GW luminosity $\dot E$~\cite{Yunes:2009ke,Chatziioannou:2012rf}. The former (latter) correspond to conservative (dissipative) corrections. Here, we take a slightly different approach and consider corrections to the GW frequency evolution $\dot f$ and the Kepler's law $r(f)$, where $r$ is the orbital separation while $f$ is the GW frequency. This is because these two quantities directly determine the amplitude and phase corrections away from GR, and hence, the final expressions are simpler than the original ones. Moreover, non-GR corrections to $\dot f$ and $r(f)$ have already been derived in the literature for many modified theories of gravity.

ppE gravitational waveform for a compact binary inspiral in the Fourier domain is given by ~\cite{Yunes:2009ke}
\begin{equation}\label{eq:2a}
\tilde{h}(f)=\tilde{h}_{\GR}(1+\alpha u^a)e^{i\delta\Psi}\,,
\end{equation}
% 
where $\tilde{h}_{\GR}$ is the gravitational waveform in GR. $\alpha u^a$ corresponds to the non-GR correction to the GW amplitude while  $\delta \Psi$ is that to the GW phase with 
\begin{equation}
u=(\pi \mathcal{M} f)^\frac{1}{3}\,.
\end{equation}
$\mathcal{M}=(m_1m_2)^{3/5}/(m_1+m_2)^{1/5}$ is the chirp mass for the binary with component masses $m_1$ and $m_2$. $u$ is proportional to the relative velocity of the binary components. $\alpha$ represents the overall magnitude of the amplitude correction while $a$ gives the velocity dependence of the correction term. In a similar manner, one can rewrite the phase correction as 
\begin{equation}\label{eq:2b}
\delta\Psi=\beta u^b\,,
\end{equation}
$\alpha$, $\beta$, $a$, and $b$ are called ppE parameters. When $(\alpha,\beta) = (0,0)$, Eq.~\eqref{eq:2a} reduces to the waveform in GR.

%In order to obtain the expressions of ppE parmeters for different modified gravity theories, we consider corrections of binary orbital seperation and the frequency evolution. These corrections come from dissipative and conservative corrections which means modification of binding energy $E$ and the rate of change of binding energy $\dot{E}$ respectively. Modification of binary orbital seperation results from conservative correction while modification of frequency evolution can come from both conservative and dissipative corrections.

As we mentioned earlier, the ppE modifications in Eq.~\eqref{eq:2a} enter through corrections to the orbital separation and the frequency evolution.
We parameterize the former as
 \begin{equation}
 \label{eq:2k}
 r=r_{\GR}(1+\gamma_r u^{c_r})\,,
 \end{equation}
where $\gamma_r$ and $c_r$ are non-GR parameters which show the deviation of the orbital separation $r$ away from the GR contribution $r_{\GR}$. To leading PN order, $r_{\GR}$ is simply given by the Newtonian Kepler's law as $r_{\GR}=\left(m/\Omega^2\right)^{1/3}$. Here $m\equiv m_1+m_2$ is the total mass of the binary while $\Omega\equiv\pi f $ is the orbital angular frequency. The above correction to the orbital separation comes purely from conservative corrections (namely corrections to the binding energy).

%Since any deviation from GR has to be small, we consider $\gamma_r$ and $c_r$ to be small as well.
%$\eta\equiv\mu/m$ is the symmetric mass ratio where $\mu=(m_1m_2)/m$ is the reduced mass of the binary.

Similarly, we parameterize the GW frequency evolution with non-GR parameters $\gamma_{\dot{f}}$ and $c_{\dot{f}}$  as
\begin{equation}\label{eq:2m}
\dot{f}=\dot{f}_{\GR}\left(1+\gamma_{\dot{f}}u^{c_{\dot{f}}}\right)\,.
\end{equation}
Here $\dot{f}_{\GR}$ is the frequency evolution in GR which, to leading PN order, is given by~\cite{cutlerflanagan,Blanchet:1995ez}
\begin{align}\label{eq:2s}
\dot{f}_{\GR}=\frac{96}{5}\pi^{8/3}\mathcal{M}^{5/3}f^{11/3}=\frac{96}{5\pi\mathcal{M}^2}u^{11}\,.
\end{align}
%$\gamma_{\dot{f}}$ and $c_{\dot{f}}$ can take different forms depending on whether the dominant correction is dissipative or conservative.
Unlike the correction to the orbital separation, the one to the frequency evolution originates corrections from both the conservative and dissipative sectors.

Below, we will derive how the ppE parameters $(\alpha, \beta, a, b)$ are given in terms of $(\gamma_r,c_r)$ and $(\gamma_{\dot{f}},c_{\dot{f}})$. We will also show how the amplitude ppE parameters $(\alpha, a)$ can be related to the phase ppE ones $(\beta,b)$ in certain cases. We will assume that non-GR corrections are always smaller than the GR contribution and keep only to leading order in such corrections at the leading PN order.

%------------------------------------------------------- 
 \subsection{Amplitude Corrections}
 
Let us first look at the amplitude corrections.  Within the stationary phase approximation~\cite{PhysRevD.62.084036,Yunes:2009yz} the waveform amplitude for the dominant quadrupolar radiation in Fourier domain is given by
 \begin{equation}\label{eq:2h1}
\tilde{\mathcal{A}}(f)=\frac{\mathcal{A}(\bar{t})}{\sqrt{\dot{f}}}\,.
\end{equation}
Here $\mathcal A$ is the waveform amplitude in the time domain while $\bar t (f)$ represents time at the stationary point. $A(\bar t)$ can be obtained by using the quadrupole formula for the metric perturbation in the transverse-traceless gauge given by~\cite{Blanchet:2002av}
 \begin{equation}\label{eq:2h2}
h^{ij}(t)\propto \frac{G}{D_L}\frac{d^2 }{d t^2}Q^{ij}\,.
 \end{equation}
Here $D_L$ is the source's luminosity distance and $Q^{ij}$ is the source's quadruple moment tensor. 

For a quasi-circular compact binary, $\tilde{\mathcal{A}}$ in Eq.~\eqref{eq:2h1} then becomes
\begin{equation}\label{eq:2e}
\tilde{\mathcal{A}}(f) \propto\frac{1}{\sqrt{\dot{f}}}\frac{G}{D_L}\mu r^2f^2 \propto \frac{r^2}{\sqrt{\dot{f}}}  \,,
\end{equation}
where $\mu$ is the reduced mass of the binary.
Substituting Eqs.~\eqref{eq:2k} and \eqref{eq:2m} into Eq.~\eqref{eq:2e} and keeping only to leading order in non-GR corrections, we find
\begin{equation}\label{eq:2n}
\tilde{\mathcal{A}}(f)=\tilde{\mathcal{A}}_{\GR} \left(1+2\gamma_ru^{c_r}-\frac{1}{2}\gamma_{\dot{f}}u^{c_{\dot{f}}}\right)\,,
\end{equation}
where $\tilde{\mathcal{A}}_{\GR} $ is the amplitude of Fourier waveform in GR. Notice that this expression is much simpler than that in the original formalism in Eq.~\eqref{eq:o2}.

Let us now show the expressions for the ppE parameters $\alpha$ and $a$ for three different cases using Eq.~\eqref{eq:2n}:

\begin{itemize}

%\subsubsection*{Dissipative Dominant Correction}
\item
\emph{Dissipative-dominated Case}

When dissipative corrections dominate, we can neglect corrections to the binary separation $(\gamma_{r} = 0)$ and Eq.~\eqref{eq:2n} reduces to
\begin{equation}
\tilde{\mathcal{A}}(f)=\tilde{\mathcal{A}}_{\GR} \left(1-\frac{1}{2}\gamma_{\dot{f}}u^{c_{\dot{f}}}\right)\,.
\end{equation}
Comparing this with the ppE waveform in Eq.~\eqref{eq:2a}, we find
\begin{equation}\label{eq:2t}
\alpha=-\frac{\gamma_{\dot{f}}}{2},\quad a=c_{\dot{f}}\,.
\end{equation}

%\subsubsection*{Conservative Dominant Correction}
\item
\emph{Conservative-dominated Case}

When conservative corrections dominate, $c_r = c_{\dot f}$ and there is an explicit relation between $\gamma_{r}$ and $\gamma_{\dot f}$. Though finding such a relation is quite involved and one needs to go back to the original ppE formalism as explained in App.~\ref{appendix}. Non-GR corrections to the GW amplitude in such a formalism is shown in Eq. \eqref{eq:o}. Setting the dissipative correction to zero, one finds 
%Correction to $\dot{f}$ is not independent of correction to the binary seperation $a$, since both dissipatve and conservative corrections contribute to the correction of $\dot{f}$. In App.~\ref{appendix}, we derived the expressions of $\dot{f}$ and ppE amplitude in Fourier space showing contribution of dissipative and conservative corrections explicitly.\st{plagiarigm alert: It's not ``our" derivation, right? What would be the best way to refer to it?} From Eq. \eqref{eq:o}, setting $B=0$ we find the ppE parameters for conservative dominant correction as
\ky{I changed $c_r$ to $a$ in the $\alpha$ expression below. I also added $c_{\dot f}$ in the $a$ expression.}
\begin{align}\label{eq:2u}
\alpha=-\frac{\gamma_r}{a}(a^2-4a-6)\,, \quad a=c_r = c_{\dot f}\,.
\end{align} 

%and
%\begin{equation}\label{eq:2x}
%a=c_r\,.
%\end{equation}

%\subsubsection*{Dissipative and Conservative at the Same Order}
\item
\emph{Comparable Dissipative and Conservative Case}

If dissipative and conservative corrections enter at the same PN order, we can set $c_r=c_{\dot{f}}$ in Eq.~\eqref{eq:2n}, which gives 
\begin{equation}
\label{eq:amp-ppE-comparable}
\alpha=2 \text{$\gamma_r $}-\frac{\text{$\gamma_{\dot{f}} $}}{2}\,, \quad a=c_{r}=c_{\dot{f}}\,.
\end{equation}
%and
%\begin{equation}
%a=c_{r}=c_{\dot{f}}\,.
%\end{equation}

\end{itemize}

%------------------------------------------------------- 
\subsection{Phase Corrections}

Next, let us study corrections to the GW phase. The phase in the Fourier domain $\Psi$ is related to the frequency evolution as~\cite{Tichy:1999pv}
\begin{equation}
\frac{d^2\Psi}{d \Omega^2}=2\frac{d t}{d\Omega}\,,
\end{equation}
which can be rewritten as
\begin{equation}
\frac{d^2\Psi}{d \Omega^2}=\frac{2}{\pi \dot{f}}\,.
\end{equation}
Substituting Eq.~\eqref{eq:2m} to the right hand side of the above equation and keeping only to leading non-GR correction, we find 
\begin{equation}\label{eq:2q}
\frac{d^2\Psi}{d \Omega^2}=\frac{2}{\pi\dot{f}_{\GR}}(1-\gamma_{\dot{f}}u^{c_{\dot{f}}})\,.
\end{equation}
%To leading PN order, $\dot{f}_{\GR}$ can be written as a function of GW frequency as \cite{VanDenBroeck:2006qu}
%\begin{equation}\label{eq:2p}
%\dot{f}_{\GR}=\frac{96}{5\pi\mathcal{M}^2}(\pi \mathcal{M}f)^{11/3}=\frac{96}{5\pi\mathcal{M}^2}u^{11}\,.
%\end{equation}
Using further Eq.~\eqref{eq:2s} to Eq.~\eqref{eq:2q} gives
\begin{equation}\label{eq:2r}
\frac{d^2\Psi}{d \Omega^2}=\frac{5}{48}\mathcal{M}^2u^{-11}(1-\gamma_{\dot{f}}u^{c_{\dot{f}}})\,.
\end{equation}
We are now ready to derive $\Psi$ and extract the ppE parameters $\beta$ and $b$. Using $\Omega = \pi f$, we can integrate Eq.~\eqref{eq:2r} twice to find 
\be
\label{eq:Psi}
\Psi = \Psi_\GR  -\frac{15 \text{$\gamma_{\dot{f}} $}}{16 (\text{$c_{\dot{f}}$}-8) (\text{$c_{\dot{f}}$}-5)} u^{c_{\dot{f}}-5}\,,
\ee
for $c_{\dot{f}} \neq 5$ and $c_{\dot{f}} \neq 8$. Here we only keep to leading non-GR correction and $\Psi_{\GR}$ is the GR contribution given in Eq.~\eqref{eq:Psi_GR} to leading PN order. Similar to the amplitude case, the above expression is much simpler than that in the original formalism in Eq.~\eqref{eq:p2}.
Comparing this with Eqs.~\eqref{eq:2a} and~\eqref{eq:2b}, we find
%The phase can be split into the GR contribution $\Psi_{\GR}$ (given in Eq.~\eqref{eq:Psi_GR} to leading PN order) and the non-GR correction $\delta\Psi$ as
%\begin{equation}\label{eq:2o}
%\Psi=\Psi_{\GR}+\delta\Psi\,,
%\end{equation}
%Substituting this into the left hand side of Eq.~\eqref{eq:2r} and comparing it with the right hand side, we find
%\begin{equation}
%\frac{b}{3}\left(\frac{b}{3}-1\right) \beta \, u^{b-6}=-\frac{5\pi}{48}\gamma_{\dot{f}}u^{c_{\dot{f}}-11}\,,
%\end{equation}
%which gives
\begin{equation}
\label{eq:2v}
\beta=-\frac{15 \text{$\gamma_{\dot{f}} $}}{16 (\text{$c_{\dot{f}}$}-8) (\text{$c_{\dot{f}}$}-5)}\,, \quad b=c_{\dot{f}}-5\,.
\end{equation}
%and
%\begin{equation}\label{eq:2v}
%\beta=-\frac{15 \text{$\gamma_{\dot{f}} $}}{16 (\text{$c_{\dot{f}}$}-8) (\text{$c_{\dot{f}}$}-5)}\,,
%\end{equation}
%for $c_{\dot{f}} \neq 5$ and $c_{\dot{f}} \neq 8$. 
The above relation is valid for all three types of corrections considered for the GW amplitude case. 

In App.~\ref{appendix}, we review $\delta\Psi$ derived in the original ppE formalism, where we show dissipative and conservative contributions explicitly. In particular, one can use Eq.~\eqref{eq:p} to find $\beta$ for all three cases separately.

%When dissipative correction dominates, we set $\gamma_r=0$ and Eq. \eqref{eq:p} gives
%\begin{equation}
%\beta=-\frac{15}{32}B\frac{1}{(4-q)(5-2q)}\eta^{-\frac{2q}{5}}\,,
%\end{equation}
%and
%\begin{equation}
%b=2q-5\,,
%\end{equation}
%with $B$ and $q$ defined by Eq.~\eqref{eq:a}.
%For conservative dominant correction, we set $B=0$ in \eqref{eq:p} which gives
%\begin{equation}\label{eq:2w}
%\beta=-\frac{15}{8}\frac{\gamma_r}{c_r}\frac{c_r^2-2c_r-6}{(8-c_r)(5-c_r)}\,,
%\end{equation}
%with
%\begin{equation}
%b=c_r-5\,.
%\end{equation}
%When dissipative and conservative correction enter at the same order, $\beta$ is given by Eq.~\eqref{eq:p} as
%\begin{align}
%\beta=-\frac{15}{8}\frac{\gamma_r}{c_r}\frac{c_r^2-2c_r-6}{(8-c_r)(5-c_r)}-\frac{15}{32}B\frac{1}{(4-q)(5-2q)}\eta^{-\frac{2q}{5}}\,,
%\end{align}
%with
%\begin{equation}
% b=2q-5=c_r-5\,.
%\end{equation}



%------------------------------------------------------- 
\subsection{Relations among ppE Parameters}

Finally, we study relations among ppE parameters. From Eqs.~\eqref{eq:2t}--\eqref{eq:amp-ppE-comparable} and~\eqref{eq:2v}, one can easily see 
\begin{equation}
b=a-5\,,
\end{equation}
which holds in all three cases considered previously. Let us consider such three cases in turn below to derive relations between $\alpha$ and $\beta$. 

\begin{itemize}

\item
\emph{Dissipative-dominated Case}

When dissipative corrections dominate, we can use Eqs.~\eqref{eq:2t} and~\eqref{eq:2v} to find $\alpha$ in terms of $\beta$ and $a$ as
\begin{equation}\label{eq:2w2}
\alpha = \frac{8}{15} (a-8)(a-5) \, \beta\,.
\end{equation}


\item
\emph{Conservative-dominated Case}

When conservative corrections dominate, we can set the dissipative correction to vanish in Eq.~\eqref{eq:p} to find 
\begin{equation}\label{eq:2w}
\beta=-\frac{15}{8}\frac{\gamma_r}{c_r}\frac{c_r^2-2c_r-6}{(8-c_r)(5-c_r)}\,, \quad b=c_r-5\,.
\end{equation}
Using this equation together with Eq.~\eqref{eq:2u}, we find
\begin{equation}
\alpha =\frac{8}{15} \frac{(8-a)(5-a)(a^2-4a-6)}{a^2-2a-6} \beta\,.
\end{equation}


\item
\emph{Comparable Dissipative and Conservative Case}

When dissipative and conservative corrections enter at the same PN order, there is no explicit relation between $\alpha$ and $\beta$. This is because $\alpha$ depends both on $\gamma_r$ and $\gamma_{\dot f}$ (see Eq.~\eqref{eq:amp-ppE-comparable}) while $\beta$ depends only on the latter (see Eq.~\eqref{eq:Psi}), and there is no relation between the former and the latter. Thus, one can rewrite $\gamma_{\dot f}$ in terms of $\beta$ and substitute into Eq.~\eqref{eq:amp-ppE-comparable} but cannot eliminate $\gamma_r$ from the expression for $\alpha$. 

\end{itemize}




%%%%%%%%%%%%%%%%%%%%%%%%%%
 \section{Example Theories}


\ky{At the beginning of each subsection below, you need to explain what each theory is, how it's different from GR, why we care about that theory, what are }

 \subsection{Scalar-Tensor Theories}
Scalar-tensor theories are one of the most well-established modified theories of gravity where at least one scalar field is introduced through a non-minimal coupling to gravity~\cite{Berti:2015itd,Chiba:1997ms,PhysRevD.6.2077}. Such theories arise naturally from the dimensional reduction of higher dimensional theories, such Kaluza-Klein theory~\cite{Fujii:2003pa,Overduin:1998pn} and string theories~\cite{polchinski1,polchinski2}.
%, and also from brane-world scenario \cite{Randall:1999vf,Randall:1999ee}. 
Scalar-tensor theories have implications in cosmology as well since they are viable candidates for accelerating expansion of our Universe \cite{Brax:2004qh,PhysRevD.73.083510,PhysRevD.62.123510,PhysRevD.66.023525,Schimd:2004nq}, structure formation \cite{Brax:2005ew}, inflation \cite{Burd:1991ns,Barrow:1990nv,Banerjee:1993ct,Clifton:2011jh}, and primordial nucleosynthesis \cite{Coc:2006rt,Damour:1998ae,Larena:2005tu,Torres:1995je}. Such theories also offer simple ways to self-consistently model possible variations in Newton's constant \cite{Clifton:2011jh}. One of the simplest scalar-tensor theories is Brans-Dicke (BD) theory, where a non-canonical scalar field is non-minimally coupled to the metric with an effective strength inversely proportional to the coupling parameter $\omega_{\BD}$ \cite{PhysRev.124.925,Scharre:2001hn}. So far the most stringent bound on the theory has been placed by the Cassini-Huygens satellite mission via Shapiro time delay measurement, which gives $\omega_{\BD}>4\times10^4$ \cite{Bertotti:2003rm}. Another class of scalar-tensor theories that has been studied extensively is Damour-Esposito-Far\`ese (DEF) gravity, which has two coupling constants $(\alpha_0,\beta_0)$ and $\alpha_0$ is directly related to $\omega_\BD$. This theory predicts nonperturbative spontaneous or dynamical scalarization phenomena for neutron stars \cite{PhysRevLett.70.2220,Barausse:2012da}. 

When scalarized NSs form compact binaries, these systems emit scalar dipole radiation that changes the orbital evolution from that in GR. Such an effect can be used to place bounds on scalar-tensor theories. For example, combining observational orbital decay results from multiple binary pulsars, the strongest upper bound on $\beta_0$ that controls the magnitude of scalarization in DEF gravity has been obtained as $\beta_0\gtrsim -4.38$ at $90\%$ confidence level \cite{Shao:2017gwu}. 
%In dynamic strong field scenario Scalar-tensor theories make predictions that are significantly different from GR predictions, which opens the possibility that such theories could be constrained by GW observation. Indeed, for Brans-Dicke  theory, it has been found that proposed LISA space interferometer could provide 10 times stronger constraint \cite{Scharre:2001hn}, a third-generation ground-based GW detector named Einstein Telescope could yield 100 times stronger constraint \cite{Zhang:2017sym}, while space-based DECIGO/BBO project could place $10^4$ times stronger constraint \cite{Yagi:2009zz} on the coupling parameter $\omega_{\BD}$ compared to the solar-system experiments.

Can BHs also possess scalar hair like NSs in scalar-tensor theories? BH no-hair theorem can be applied to many of scalar-tensor theories that prevents BHs to acquire scalar charges~\cite{Hawking:1972qk,Bekenstein:1995un,Sotiriou:2011dz,Hui:2012qt,Maselli:2015yva} including BD and DEF gravity, though exceptions exist, such as EdGB gravity~\cite{Yunes:2011we,Sotiriou:2013qea,Sotiriou:2014pfa,Silva:2017uqg,Doneva:2017bvd} that we explain in more detail in the next subsection. On the other hand, if the scalar field cosmologically evolves as a function of time, BHs can acquire scalar charges, known as the BH miracle hair growth~\cite{Jacobson:1999vr,Horbatsch:2011ye} (see also~\cite{Healy:2011ef,Berti:2013gfa} for related works). 

%So how the BHs in scalar-tensor theories are different from GR BHs? One prominent non-GR feature of Scalar-tensor BHs is that they can evade no-hair theorem and acquire scalar hair supported by time-dependent boundary conditions \cite{Berti:2013gfa}. Such time-dependence could arise either from cosmological evolution, or due to the slow motion of the black hole within the asymptotic spatial gradient in the scalar field \cite{Horbatsch:2011ye}. Because of the scalar charge, inspiriling binary black holes can emit dipole scalar radiation, given that the mass of the scalar particle is smaller than the orbital period ($m \ll \Omega$)\cite{Horbatsch:2011ye}. 

Scalar hair of compact objects in a binary produce scalar dipole radiation, which changes the evolution of the binary and modifies the gravitational waveform from that in GR. Using the orbital decay rate of compact binaries in scalar-tensor theories in~\cite{Freire:2012mg,Wex:2014nva}, one can read off non-GR corrections to $\dot f$ as
\be
\gamma_{\dot f} = \frac{5}{96} \eta ^{2/5}(\alpha_1-\alpha_2)^2\,,
\ee
with $c_{\dot f} = -2$.
Given that the leading correction to the waveform is the dissipative one in scalar-tensor theories, one can use Eq.~\eqref{eq:2v} to derive the PPE phase correction as
\be\label{eq:betaST}
\beta_{\ST}=-\frac{5}{7168}\eta ^{2/5}(\alpha_1-\alpha_2)^2\,,
\ee
with $b=-7$. Here $\alpha_A$ represents the scalar charge of the $A$th binary component.
Using further Eq.~\eqref{eq:2w2}, one finds the amplitude correction as
\be\label{eq:alphaST}
\alpha_{\ST}=-\frac{5}{192}\eta ^{2/5}(\alpha_1-\alpha_2)^2\,,
\ee
with $a=-2$. 


The scalar charges $\alpha_A$ depend on specific theories and compact objects. For example, in situations where the BH no-hair theorem~\cite{Hawking:1972qk,Bekenstein:1995un,Sotiriou:2011dz} applies, $\alpha_A = 0$. On the other hand, if the scalar field is evolving cosmologically, BHs undergo \emph{miracle hair growth}~\cite{Jacobson:1999vr} and acquire scalar charges given by~\cite{Horbatsch:2011ye}
\be
\alpha_A = 2 \, m_A \, \dot \phi\, [1+(1-\chi_A^2)^{1/2}]\,,
\ee
where $\dot{\phi}$ is the growth rate of the scalar field, $m_A$ is individual masses and $\chi_A$ is the magnitude of the spin angular momentum of the $\mathit{A}\text{th}$ body normalized by its mass squared. The PPE phase parameter $\beta$ for binary BHs in such a situation was derived in~\cite{Yunes:2016jcc}. Another well-studied example is Brans-Dicke theory, where one can replace $(\alpha_1-\alpha_2)^2$ in Eqs.~\eqref{eq:betaST} and~\eqref{eq:alphaST} as $2 (s_1-s_2)^2/(2+\omega_\BD)$~\cite{Freire:2012mg}. Here $s_A$ is the sensitivity of the $A$th body and roughly equals to its compactness (0.5 for BHs and $\sim 0.2$ for NSs). PPE parameters in this theory has been found in~\cite{Chatziioannou:2012rf}. Scalar charges and PPE parameters in generic screened modified gravity have recently been derived in~\cite{Zhang:2017srh,Liu:2018sia}.

\ky{I moved this paragraph here since it makes more sense to discuss GW bounds after showing PPE expressions.}
The phase correction in Eq.~\eqref{eq:betaST} has been used to derive current and future projected bounds with GW interferometers. Regarding the former, GW150914 and GW151226 do not place any meaningful bounds on $\dot \phi$~\cite{Yunes:2016jcc}. On the other hand, by detecting GWs from BH-NS binaries, aLIGO and Virgo with their design sensitivities can place bounds that are stronger than the above binary pulsar bounds from dynamical scalarization for certain equations of state and NS mass range \cite{Shibata:2013pra,Taniguchi:2014fqa,Sampson:2014qqa,Shao:2017gwu}\footnote{One needs to multiply Eq.~\eqref{eq:betaST} by a step-like function to capture the effect of dynamical scalarization.}. Einstein Telescope, a third generation ground-based detector, can yield constraints on BD theory from BH-NS binaries that are 100 times stronger the current bound~\cite{Zhang:2017sym}. Projected bounds with future space-borne interferometers, such as DECIGO, can be as large as four orders of magnitude stronger than current bounds~\cite{Yagi:2009zz}, while those with LISA may not be as strong as the current bound~\cite{Berti:2004bd,Yagi:2009zm}. 



%While Eqs. \eqref{eq:betaST} and \eqref{eq:alphaST} are applicable for both BH and NS binaries, for the simplest case of scalar-tensor theories containing a single scalar field, we can write an expression of $\beta$ for BH binaries as  \cite{Jacobson:1999vr,Horbatsch:2011ye,Yunes:2016jcc}
% \begin{equation}\label{eq:betaBH}
% \beta_{\dot{\phi}}=-\frac{5}{1792}\dot{\phi}^2\eta^{\frac{2}{5}}(m_1\tilde s_1^{\ST}-m_2\tilde s_2^{\ST})^2\,,
% \end{equation}
%where $\dot{\phi}$ is the growth rate of scalar field, $m_A$ is the individual mass, and $\tilde s_A^{\ST}\equiv[1+(1-\chi_A^2)^{1/2}]/2$ are the spin-dependent factors of BH scalar charges \cite{Horbatsch:2011ye} where $\chi_A$ is the magnitude of the spin angular momentum of the $\mathit{A}\text{th}$ body normalized by its mass squared.
%As for the specific case of NS binaries, ppE phase correction in BD theory is given by
%\be\label{eq:betaNS}
%\beta_{\BD}=-\frac{5}{3584}\frac{\eta ^{2/5}}{2+\omega_\BD} (s_1^\BD-s_2^\BD)^2\,,
%\ee
%where $s_A^\BD$ are sensitivities. For both Eq. \eqref{eq:betaBH} and Eq. \eqref{eq:betaNS}, correction to the amplitude can calculated from Eq. \eqref{eq:2w2}.
%\st{Although I said ``NS binaries", I think Eq. \eqref{eq:betaNS} applies for NS-NS, NS-WD, and WD-WD binaries. I am not sure though.}


In some of scalar-tensor theories, the gravitational constant $G$ is a function of the scalar field, and thus it can vary with time if the scalar field is cosmologically evolving. As we will see in Sec.~\ref{gdot}, such an effect enters at -4PN order in gravitational waveforms, and hence, the effect of scalar dipole radiation discussed here (entering at -1PN order) becomes subdominant in terms of a PN order counting. However, we note that these effects are completely independent and should be treated separately. For example, in compact stellar binaries the latter -1PN effect can dominate the former -4PN effect if the scalar field is almost time independent.

%It is worth mentioning that dissipative and conservative corrections enter at the same order for scalar-tensor theories with time varying Newton's constant. For such theories Eq. \eqref{eq:betaST} and \eqref{eq:alphaST} do not give the leading order non-GR corrections, and One should follow the formalism of Sec.~\ref{gdot} in order to calculate ppE parameters.
%\st{In that case may be we should not give the example of BD theory above.........}
 
  \subsection{EdGB Gravity}
EdGB gravity is a well-known extension of GR, which emerges naturally in the framework of low-energy ee string theories  and gives one of the simplest viable high-energy modifications to GR \cite{Moura:2006pz,Pani:2009wy}.ffectiv It also arises as a special case of Horndeski gravity \cite{Zhang:2017unx,Berti:2015itd}, which is the most generic scalar-tensor theory with at most second-order derivatives in the field equations. One obtains the EdGB action by adding a quadratic-curvature term to the Einstein-Hilbert action, where the scalar field (dilaton) is non-minimally coupled to the Gauss-Bonnet term with a coupling constant $\bar{\alpha}_\EDGB$ \cite{Kanti:1995vq}. A stringent upper bound on such a coupling constant has been placed using the orbital deca measurement of a BH low-mass X-ray binary (LMXB) as $\sqrt{|\bar{\alpha}_\EDGB|} < 1.9\times10^5$ cm~\cite{Yagi:2012ygp}. A similar upper bound has been placed from the existence of BHs~\cite{Pani:2009wy}. Equation-of-state-dependent bounds from the maximum mass of neutron stars have also been derived in~\cite{pani-EDGB-NS}. 


BHs in EdGB theory are of particular interest since they are fundamentally different from their GR counterparts. Perturbative but analytic solutions are available for static~\cite{Mignemi:1992nt,Mignemi:1993ce,Yunes:2011we,Sotiriou:2014pfa} and slowly rotating EdGB BHs~\cite{Pani:2011gy,Ayzenberg:2014aka,Maselli:2015tta} while numerical solutions have been found for static~\cite{Kanti:1995vq,Torii:1996yi,Alexeev:1996vs} and rotating~\cite{Pani:2009wy,Kleihaus:2011tg,Kleihaus:2014lba} BHs. One of the important reasons for considering BHs in EdGB is that BHs acquire scalar monopole charge~\cite{Yagi:2011xp,Sotiriou:2014pfa,Berti:2018cxi,Prabhu:2018aun} while ordinary stars such as neutron stars do not if the scalar field is coupled linearly to the Gauss-Bonnet term in the action~\cite{Yagi:2011xp,Yagi:2015oca}. This means that binary pulsars are inefficient to constrain the theory, and one needs systems such as BH-LMXBs~\cite{Yagi:2012gp} or BH/pulsar binaries~\cite{Yagi:2015oca} to have better probes on the theory.

We now show the expressions of ppE parameters for EdGB gravity. The scalar monopole charge of EdGB BHs in  generates scalar dipole radiation, which leads to an earlier coalescence of BH binaries compared to GR. Such scalar radiation modifies the GW phase with the ppE parameter $\beta$ as~\cite{Yunes:2016jcc,Yagi:2011xp}
\begin{equation}
\label{eq:beta-EdGB}
 \beta_\EDGB=-\frac{5}{7168}\zeta_\EDGB\frac{(m_1^2\tilde s_2^\EDGB-m_2^2\tilde s_1^\EDGB)^2}{m^4\eta^{18/5}}\,.
 \end{equation}
 Here, $\zeta_\EDGB\equiv 16 \pi \bar{\alpha}_\EDGB^2/m^4$ is the dimensionless EdGB coupling parameter and $\tilde s_{A}^\EDGB$ are the spin-dependent factors of the BH scalar charges given by $\tilde s_{A}^\EDGB\equiv 2(\sqrt{1-{\chi_A}^2}-1+{\chi_A}^2)/{\chi_A}^2~$ \cite{Berti:2018cxi,Prabhu:2018aun}. In EdGB gravity, the leading order correction to the phase enters through the correction of GW eneregy flux, and hence the theory corresponds to a dissipative-dominated case. We can then use Eq. \eqref{eq:2w2} to calculate the amplitude ppE parameter $\alpha_\EDGB$ from $\beta_\EDGB$ as  
 \begin{equation}
 \alpha_\EDGB=-\frac{5}{192}\zeta_\EDGB\frac{(m_1^2 \tilde s_2^\EDGB-m_2^2 \tilde s_1^\EDGB)^2}{m^4\eta^{18/5}}\,.
 \end{equation}
 
 \ky{I added this paragraph.}
 One can use the phase correction in Eq.~\eqref{eq:beta-EdGB} to derive bounds on EdGB gravity with current~\cite{Yunes:2016jcc} and future~\cite{Yagi:2012gp} GW observations. Similar to the scalar-tensor theory case, current binary BH GW events do not allow us to place any meaningful bounds on the theory. Future second- and third-generation ground-based detectors and LISA can place bounds that are comparable to current bounds from low-mass X-ray binaries~\cite{Yagi:2012gp}. On the other hand, DECIGO has the potential to go beyond the current bounds by three orders of magnitude.
 
 %----------------------------------
  \subsection{dCS Gravity}
  
DCS gravity is described by Einstein-Hilbert action with a dynamical (pseudo)scalar field which is non-minimally coupled to the Pontryagin density with a coupling constant $\bar{\alpha}_{\DCS}$~\cite{Alexander:2009tp,Jackiw:2003pm}. Similar to EdGB gravity, dCS gravity arises as an effective field theory from the compactification of  heterotic string theory \cite{GREEN1984117,McNees:2015srl}. Such a theory is also important in the context of particle physics \cite{Alexander:2009tp,Mariz:2004cv,MARIZ2008312,PhysRevD.78.025029}, loop quantum gravity  \cite{PhysRevD.80.104007,Taveras:2008yf}, and inflationary cosmology \cite{Weinberg:2008hq}. Demanding that the critical length scale (below which higher curvature corrections beyond quadratic order cannot be neglected in the action) has to be smaller than the scale probed by table-top experiments, one finds $\sqrt{|\bar{\alpha}_{\DCS}|} < \mathcal{O}(10^8 \text{km})$ \cite{Yagi:2012ya}. Similar constraints have been placed from measurements of the frame-dragging effect by Gravity Probe B and LAGEOS satellites \cite{AliHaimoud:2011fw}.

%Since any spherically symmetric solution of GR is a solution of dCS gravity as well \cite{Yunes:2007ss,Berti:2015itd}, it is difficult to distinguish it from GR. Also a large-curvature environment is required to have significant correction from dCS gravity, which makes binary black hole mergers a promising sector for this purpose. A projected bound on $\bar{\alpha}_{\DCS}$ has been calculated in Ref. \cite{Yagi:2012vf} from the numerical analysis of BBH mergers, compared to solar system experiments which gives six to seven orders of magnitude stronger constraint on $\bar{\alpha}_{\DCS}$. Also a numerical study of Extreme-Mass-Ratio Inspirals (EMRI) has been done in Ref. \cite{Canizares:2012is} and compared solar system bound it gives three orders of magnitude stronger constraint. Since equal-mass and equal-spin binaries do not give any correction for dCS gravity, recent GW observation events GW150914 and GW151226 by Advanced LIGO detectors could not place any constraint on this theory \cite{Yunes:2016jcc}. 

We now derive the expressions of ppE parameters for dCS gravity. While BHs in EdGB gravity possess scalar monopole charges, BHs in dCS gravity possess scalar dipole charges which induce scalar quadrupolar emission~\cite{Yagi:2011xp}. On the other hand, scalar dipole charges induce a scalar interaction force between two BHs. Each BH also acquires a modification to the quadrupole moment away from the Kerr value. All of these modifications result in both dissipative and conservative corrections entering at the same order in gravitational waveforms. For spin-aligned binaries\footnote{See recent works~\cite{Loutrel:2018rxs,Loutrel:2018ydv} for precession equations in dCS gravity.}, corrections to the Kepler's law and frequency evolution in dCS gravity are given in~\cite{Yagi:2012vf} within the slow-rotation approximation for BHs, from which we can derive
\begin{align}\label{dcs:gamma_r}
\gamma_r=&\frac{25}{256}\eta^{-9/5}\zeta_{\DCS}\chi_1 \chi_2 \nonumber \\ 
&-\frac{201}{3584}\eta^{-14/5}\zeta_{\DCS} \left(\frac{m_1^2}{m^2}\chi_2^2 +\frac{m_2^2}{m^2}\chi_1^2\right)\,,
\end{align}
with $c_r=4$, and
 \begin{align}\label{dcs:gamma_fdot}
\gamma_{\dot{f}}=& \frac{38525}{39552}\eta^{-9/5}\zeta_{\DCS}\chi_1 \chi_2 \nonumber \\
&-\frac{309845}{553728 }\eta^{-14/5}\zeta_{\DCS} \left(\frac{m_1^2}{m^2}\chi_2^2+\frac{m_2^2}{m^2}\chi_1^2\right)\,,
 \end{align}
with $c_{\dot{f}}=4$. Here $\zeta_{\DCS}=16\pi \bar{\alpha}_{\text{dCS}}^2/m^4$ is the dimensionless coupling constant.
%$m_A$ is the individual mass, $\chi_A =\left | S_{A}^{i} \right |/m_A$ is the dimensionless Kerr spin parameter where $S_{A}^{i}$ is the spin angular momentum vector, all relative to the $\mathit{A}\text{th}$ BH. 
Using Eqs. \eqref{dcs:gamma_r} and \eqref{dcs:gamma_fdot} in Eqs.~\eqref{eq:2u} and \eqref{eq:2v} respectively, one finds
 
 \begin{align}
 \alpha_{\DCS}=&\frac{185627}{1107456}\eta ^{-14/5}\zeta_{\DCS} \left[-2 \delta_m \chi_a \chi_s \right. \nonumber\\ 
 &\left. +\left(1-\frac{53408 \eta }{14279}\right) \chi_a^2+\left(1-\frac{3708 \eta }{14279}\right) \chi_s^2\right]\,,
 \end{align}
with $a=4$, and
 \begin{align}
 \beta_{\DCS}=&\frac{1549225}{11812864}\eta ^{-14/5} \zeta_{\DCS} \left[-2 \delta_m \chi_a \chi_s \right.\nonumber\\ 
 &\left. +\left(1-\frac{16068 \eta }{61969}\right)\chi_a^2 +\left(1-\frac{231808 \eta }{61969}\right)\chi_s^2\right]\,,
 \end{align}
with $b=-1$. Here $\chi_{s,a}=(\chi_1 \pm \chi_2 )/2$ are the symmetric and antisymmetric combinations of dimensionless spin parameters 
%\footnote{$\chi_s$ is different from $\chi_{\text{eff}}\equiv (a_{||1}+a_{||2})/m$ in \cite{TheLIGOScientific:2016wfe}.} 
%with $a_{||A}$ representing the projection of the (dimensional) spin vector $\vec{a}_A$ onto the unit orbital angular momentum vector, 
and $\delta_m=(m_1-m_2)/m$ is the fractional difference in masses relative to the total mass.

\ky{I've moved this paragraph here.}
Can GW observations place stronger bounds on the theory? Current GW observations do not allow us to put any meaningful bounds on dCS gravity~\cite{Yunes:2016jcc} (see also~\cite{Yagi:2017zhb}). However, future observations have potential to place bounds on the theory that are six to seven orders of magnitude stronger than current bounds~\cite{Yagi:2012vf}. Such stronger bounds can be realized due to relatively strong gravitational field and large spins that source the pseudoscalar field. Measuring GWs from extreme mass ratio inspirals with LISA can also place bounds that are three orders of magnitude stronger than current bounds~\cite{Canizares:2012is}. 


%---------------------------- 
 \subsection{Einstein-$\AE$ther  and Khronometric Theory}

In this section, we study two example theories that break Lorentz invariance in the gravity sector, namely EA and khronometric theory. Lorentz-violating theories of gravity are candidates for low-energy descriptions of quantum gravity \cite{Blas:2014aca,Horava:2009uw}. Although Lorentz-violation in the gravity sector has not been as stringently constrained as that in the matter sector~\cite{Mattingly:2005re,Jacobson:2005bg,Liberati:2013xla}, and several mechanisms exist that prevents percolation of the former to the latter~\cite{Liberati:2013xla,Pospelov:2010mp}.
 
 Einstein-$\AE$ther (EA) theory is a vector-tensor theory of gravity, where along with the metric, a spacetime is endowed with a dynamical timelike unit vector ($\AE$ther) field \cite{Jacobson:2000xp,Jacobson:2008aj}. Such a vector field specifies a particular rest frame at each point in spacetime, and hence breaks the local Lorentz symmetry. The amount of Lorentz violation is controlled by four coupling parameters 
$(c_1,c_2,c_3,c_4)$. EA theory preserves diffeomorphism invariance and hence is a Lorentz-violating theory without abandoning the framework of GR \cite{Jacobson:2008aj}. 
%Furthermore, Lorentz-violating theories are candidates of low-energy descriptions of quantum gravity, since a number of approaches suggest that Lorentz symmetry might be broken in quantum gravity \cite{Blas:2014aca,Horava:2009uw}. 
%The presence of a Lorentz-violating vector field can have significant impact on cosmology, such as the renormalization of Newton's constant \cite{Carroll:2004ai},  and changing purtabtions in early universe which can leave imprint on CMB \cite{Kanno:2006ty,Lim:2004js,Li:2007vz}.
%
%Lorentz-violating theories predict interesting non-GR phenomena which include extra gravitational degrees of freedom. In particular, 
Along with the spin-2 gravitational perturbation of GR, the theory predicts the existence of spin-1 and spin-0 perturbations \cite{Foster:2006az,Jacobson:2004ts,PhysRevD.76.084033}. Such perturbation modes propagate at speeds that are functions of coupling parameters $c_i$, and in general differ from the speed of light \cite{Jacobson:2004ts}. 
 
%Static and spherically symmetric NS solutions exist in EA theory \cite{Eling:2007xh,Eling:2006df}, as well as solutions representing isolated non-spining NSs moving slowly relative to $\AE$ther \cite{Yagi:2013ava}. For a certain parameter range and depending on the nuclear equation of state, such solutions lead to lower maximum neutron star masses, and larger surface redshifts for a particular mass \cite{Eling:2007xh}. As for the BHs, the requirement of a causal black hole implies that along with the metric horizon, BHs in EA theory should be furnished with spin-0, spin-1, and spin-2 horizons \cite{Eling:2006ec}. Depending on the propagation speed of the corresponding field, these horzions lie at different locations \cite{Berti:2015itd,Eling:2006ec}. Although time-independent, sphercially symmetric black hole solutions have been obtained numerically, rotating black hole solutions in EA theory are yet to be found \cite{Eling:2006ec}.
 
 
Khronometric theory is a variant of EA theory, where the $\AE$ther field is restricted to be hypersurface-orthogonal. Such a theory arises as a low-energy limit of Ho\v{r}ava gravity, a power-counting renormalizable quantum gravity model with only spatial diffeomorphism invariance \cite{Blas:2009qj,Berti:2015itd,Horava:2009uw,Nishioka:2009iq,Visser:2009fg}. The amount of Lorentz violation in the theory is controlled by three parameters, $(\bar \alpha_{\KG}, \bar \beta_{\KG}, \bar \lambda_{\KG})$\footnote{We use barred quantities for coupling constants and $\bar \alpha_{\KG}, \bar \beta_{\KG}$ are not to be confused with PPE parameters $(\alpha_{\KG}, \beta_{\KG})$.}. Unlike EA theory, spin-1 propagating modes are absent in khronometric theory.

%The restriction of hyepersurface-orthogonality implies that instead of four coupling parameters, the theory can be expressed as a function of three coupling parameters. Similar to EA theory, khronometric theory has only two independent parameters ($\bar \beta_{\KG}$,$\bar \lambda_{\KG}$), and the third parameter can be expressed in terms of the former two. Although most of the predictions of EA theory are carried over to khronometric theory, there are few exceptions. Because of the hypersurface-orthogonality constraint, there is no propagating gravitational vector mode in khronometric theory. Spherically symmetric solutions for khronometric theory are identical to those of EA theory, hence the properties of non-rotating  NS in EA theory also applies to khronometric theory \cite{Berti:2015itd}. As for the BHs, the spin-1 horizon for the spin-1 field is absent, since there is no propagating vector mode \cite{Berti:2015itd}.

Most of parameter space in EA and khronometric theory have been constrained stringently from current observations and theoretical requirements. Using the measurement of the arrival time difference between GWs and electromagnetic waves in GW170817, the difference in the propagation speed of GWs away from the speed of light has been constrained to be less than $\sim 10^{-15}$~\cite{TheLIGOScientific:2017qsa,Monitor:2017mdv}. Such a bound can be mapped to bounds on Lorentz-violating gravity as $|c_1 + c_3| \lesssim 10^{-15}$~\cite{Oost:2018tcv} and $|\beta | \lesssim10^{-15}$~\cite{Gumrukcuoglu:2017ijh}\footnote{Such bounds are consistent with the prediction in~\cite{Hansen:2014ewa} based on~\cite{Nishizawa:2014zna}.}. Imposing further constraints from solar system experiments~\cite{Bailey:2006fd,Foster:2005dk,Will:2005va}, Big Bang nucleosynthesis~\cite{Audren:2013dwa} and theoretical constraints such as the stability of propagating modes, positivity of their energy density \cite{Eling:2005zq} and the absence of gravitational Cherenkov radiation\cite{Elliott:2005va},  allowed regions in the remaining parameter space have been derived for EA~\cite{Oost:2018tcv} and khronometric~\cite{Gumrukcuoglu:2017ijh} theory. Binary pulsar bounds on these theories were studied in~\cite{Yagi:2013ava,Yagi:2013qpa} before the discovery of GW170817, within a parameter space that is different from allowed regions in~\cite{Oost:2018tcv,Gumrukcuoglu:2017ijh}.

%Both EA and khronometric theory predict two (dimensionless) preferred-frame PPN parameters in weak field \cite{Foster:2005dk}, which can be constrained  from solar system experiments to very small values of $|\alpha_1| \lesssim 10^{-4}$ (from  orbital polarization effect bounded by lunar laser ranging and binary pulsar observations) and $|\alpha_2| \lesssim 10^{-7}$ (from a spin precession effect bounded by the alignment of the solar spin with the ecliptic) \cite{Bailey:2006fd,Foster:2005dk,Will:2005va}. More relativistic Lorentz-violating effects are controlled by ($c_+,c_-$) and ($\bar \beta_{\KG},\bar \lambda_{\KG}$) in EA theory and in khronometric theory respectively \cite{Yagi:2013qpa}. Stringent constraints on these parameters have been obtained by binary pulsar observations \cite{Yagi:2013ava,Yagi:2013qpa} and cosmological observations \cite{Audren:2014hza,Audren:2013dwa} as $c_+ \lesssim 0.03$, $c_- \lesssim 0.03$ for EA theory, and, $\bar \beta_{\KG} \lesssim 0.005$ and $\bar \lambda_{\KG} \lesssim 0.1$ for khronometric theory \cite{Hansen:2014ewa}. Further constraints come from the requirement that Big Bang nucleosynthesis elemental abundances are in agreement with the observations, and such constraints are much stronger for khronometric theory than that of EA theory \cite{Audren:2013dwa}. In addition to the observational constraints, theoretical constraints come from the requirement of positive energy \cite{Eling:2005zq} and the absence of gravitational Cherenkov radiation \cite{Elliott:2005va}.
% 
 
 

Let us now derive PPE parameters in EA and khronometric theories. The propagation of scalar and vector modes is responsible for dipole radiation and loss of angular momentum in binary systems, which increase the amount of orbital decay rate. 
Regarding EA theory, the PPE phase correction is given by \cite{Hansen:2014ewa} \ky{I've set $G_N=1$.}
 \begin{align}
 \label{eq:phase-EA}
 \beta_{\EA}=&-\frac{5}{3584}\eta ^{2/5} (s_1^\EA-s_2^\EA)^2\nonumber\\
 & \times \frac{(c_{14}-2) w_0^3-w_1^3}{c_{14} w_0^3 w_1^3 [(1-s_1^\EA) (1-s_2^\EA)]^{4/3}}\,,
 \end{align}
with $b=-7$. Here $w_s$ is the propagation speed of the spin-$s$ mode in EA theory given by~\cite{Jacobson:2008aj}  
\ba
w_0^2 &=& \frac{(2-c_{14}) c_{123}}{(2+3c_2+c_{+}) (1-c_{+}) c_{14}}\,, \\
w_1^2 &=& \frac{2 c_1 - c_{+} c_{-}}{2(1-c_{+}) c_{14}}\,, \\
w_2^2 &=& \frac{1}{1 - c_+}\,,
\ea
with
\be
c_{14}\equiv c_1+c_4\,, \quad c_{\pm} \equiv c_1 \pm c_3\,, \quad c_{123} \equiv c_1 + c_2 + c_3\,. 
\ee
$s_A$ in Eq.~\eqref{eq:phase-EA} is the sensitivity of the $A$-th body and has been calculated only for NSs~\cite{Yagi:2013ava,Yagi:2013qpa}.
Given that the leading order correction in EA theory arises from the dissipative sector \cite{Hansen:2014ewa}, we can use Eq. \eqref{eq:2w2} to find the ppE amplitude correction as\footnote{Eqs.~\eqref{eq:amp-EA} and~\eqref{eq:amp-KG} correct errors in the first version of~\cite{Hansen:2014ewa}.}
 \begin{align}
 \label{eq:amp-EA}
 \alpha_{\EA}=&-\frac{5}{96}\eta ^{2/5} (s_1^\EA-s_2^\EA)^2 \nonumber\\
 & \times  \frac{(c_{14}-2) w_0^3-w_1^3}{c_{14} w_0^3 w_1^3 [ (1-s_1^\EA) (1-s_2^\EA)]^{4/3}}\,,
 \end{align}
 with $a=-2$.  Similar to EA theory, PPE parameters in khronometric theory is given by \cite{Hansen:2014ewa}
 \begin{align}
 \beta_{\KG}=&-\frac{5}{3584}\eta ^{2/5}\sqrt{\bar \alpha_{\KG}}\frac{(s_1^\KG-s_2^\KG)^2}{[(1-s_1^\KG) (1-s_2^\KG)]^{4/3}} \nonumber\\
 & \times \left[\frac{(\bar \beta_{\KG}-1)(2+\bar \beta_{\KG}+3\bar \lambda_{\KG})}{(\bar \alpha_{\KG}-2)(\bar \beta_{\KG}+\bar \lambda_{\KG})}\right]^{3/2}\,,
 \end{align}
 with $b=-7$, and
 \begin{align}
 \label{eq:amp-KG}
 \alpha_{\KG}=&-\frac{5}{96} \eta ^{2/5}\sqrt{\bar \alpha_{\KG}}\frac{(s_1^\KG-s_2^\KG)^2}{[(1-s_1^\KG) (1-s_2^\KG)]^{4/3}} \nonumber\\
 & \times \left[\frac{(\bar \beta_{\KG}-1)(2+\bar \beta_{\KG}+3\bar \lambda_{\KG})}{(\bar \alpha_{\KG}-2)(\bar \beta_{\KG}+\bar \lambda_{\KG})}\right]^{3/2}\,,
 \end{align}
 with $a=-2$.
 
Above corrections to the gravitational waveform can be used to compute current and projected future bounds on the theories with GW observations, provided one knows what sensitivities are for compact objects in binaries. Unfortunately, such sensitivities have not been calculated for BHs, and hence, one cannot derive bounds on the theories from recent binary BH merger events. Instead, Ref.~\cite{Yunes:2016jcc} used the next-to-leading 0PN correction that is independent of the sensitivities and derived bounds from GW150914 and GW151226, though such bounds are weaker than those from binary pulsar observations~\cite{Yagi:2013ava,Yagi:2013qpa}. On the other hand, Ref.~\cite{Hansen:2014ewa} includes both the leading and next-to-leading corrections to the waveform and estimate projected future bounds with GWs from binary NSs. The authors found that bounds from second-generation ground-based detectors are less stringent that existing bounds even with their design sensitivities. However, third-generation ground-based ones and space-borne interferometers can place constraints that are comparable, and in some cases, stronger than current bounds approximately by a factor of 2 \cite{Hansen:2014ewa}.
  
%How stringent are the constraints on Lorentz-violation which are attainable from GW observations? Provided that the GW signal is from non-spinning NS binaries, numerical study shows that the second-generation detectors alone cannot place bounds on Lorentz violation that are more stringent than the current binary pulsar bounds \cite{Hansen:2014ewa}. However, third-generation ground and space-based detectors would be able to place constraints that are comparable, and in some cases, stronger than current bounds approximately by factor of 2 \cite{Hansen:2014ewa}. If the GW signal has a coincident electromagnetic counterpart, even when using  the second generation detector, constraints on  $c_+$ and $\bar \beta_{\KG}$ can be obtained which are at least 10 orders of magnitude more stringent than the current binary pulsar constraint \cite{Hansen:2014ewa}.  Indeed, recent gravitational-wave event GW170817 \cite{TheLIGOScientific:2017qsa} and gamma-ray burst GRB 170817A \cite{Monitor:2017mdv} placed constraints on $c_+$ and $\beta_{\KG}$ that are 13 orders of magnitude stronger than the binary pulsar constraints \cite{Gumrukcuoglu:2017ijh,Oost:2018tcv}. Projected constraints in black hole binaries are not available, since such analysis requires the black hole sensitivites beforehand, which are yet to be obtained in Lorentz-violating theories \cite{Hansen:2014ewa}.

%----------------------------------
 \subsection{Noncommutative Gravity}
Although the concept of nontrivial commutation relations of the spacetime coordinates is quite old \cite{Snyder:1946qz,Snyder:1947nq}, the idea has revived recently with the development of noncommutative geometry \cite{connes1985non,connes1995noncommutative,1987117,Landi:1997sh,Woronowicz:1987vs}, and the emergence of noncommutative structure of spacetime in a specific limit of string theory \cite{WITTEN1986253,Seiberg:1999vs}. Quantum field theories on noncommutative spacetime have been studied extensively as well \cite{Douglas:2001ba,Rivelles:2002ez,Szabo:2001kg}. In the simplest model of  noncommutative gravity the spacetime coordinates are promoted to operators, which satisfy canonical commutation relation:

\be
\left[\hat{x}^{\mu},\hat{x}^{\nu}\right]=i \theta^{\mu\nu}\,,
\ee

where $\theta^{\mu\nu}$ is a real constant antisymmetric tensor. In ordinary quantum mechanics Planck's constant $\hbar$ measures the quantum fuzziness of phase space coordinates, in a similar manner $\theta^{\mu\nu}$ introduces a new fundamental scale which measures the quantum fuzziness of spacetime coordinates \cite{Kobakhidze:2016cqh}.

%In the background of an anti-symmetric B-field, quantized spacetime represent  the low-energy field-theoretic limit of string %theory \cite{Ardalan:1998ce,Seiberg:1999vs}. 


In order to obtain stringent constraints on the scale of noncommutativity, low-energy experiments are advantageous over the high-energy ones \cite{Carroll:2001ws,Mocioiu:2000ip}. Low-energy precision measurements such as clock-comparison experiments with nuclear-spin-polarized $_{}^{9}\textrm{Be}^+$ ions \cite{PhysRevLett.54.2387} give the constraint on noncommutative scale as $1/\sqrt{\theta}\gtrsim 10$ TeV \cite{Carroll:2001ws}. Similar bound comes from the measurement of Lamb shift \cite{PhysRevLett.86.2716}. Another speculative bound is derived from the analysis of atomic experiments which is 10 orders of magnitude stronger \cite{Berglund:1995zz,Mocioiu:2000ip}. Study of inflationary observables using CMB data from PLANCK gives the lower bound on the energy scale of noncommutativity as  19 TeV \cite{calmet2015inflation,PhysRevD.91.083503}.




There are several formulations of noncommutative gravity \cite{Aschieri:2005yw,Aschieri:2005zs,Calmet:2005qm,Chamseddine:2000si,Kobakhidze:2006kb,Szabo:2006wx}, and the first order noncommutative correction vanishes in all of them \cite{Calmet:2006iz,Mukherjee:2006nd}, making the leading order correction second order in noncommutative scale. On the other hand, first order  correction can appear in gravity-matter ineraction \cite{Kobakhidze:2007jn,Mukherjee:2006nd}. Hence it suffices to ignore corrections to the pure gravity sector and consider only corrections to the matter sector (i.e., energy-momentum tensor) \cite{Kobakhidze:2016cqh}. Making corrections to classical matter source and following effective field theory approach, expressions of energy and GW luminosity for quasicircular black hole binaries have been derived in Ref. \cite{Kobakhidze:2016cqh}, which give the correction to frequency evolution as
\ba \label{noncom:gamma_fdot}
\gamma_{\dot{f}}=\frac{5}{4}\eta ^{-4/5}(2 \eta -1)\Lambda^2\,,
\ea
with $c_{\dot{f}}=4$. Angular frequency with leading order non-GR correction is also given in Ref. \cite{Kobakhidze:2016cqh}, from which the correction to binary seperation can be derived as
\be \label{noncom:gamma_r}
\gamma_r=\frac{1}{8}\eta ^{-4/5}(2 \eta -1)\Lambda^2\,,
\ee
with $c_r=4$. Dissipative and conservative corrections are comparable in noncommutativ gravity, hence using Eqs. \eqref{noncom:gamma_fdot} and \eqref{noncom:gamma_r} in Eq.~\eqref{eq:amp-ppE-comparable} one finds
\begin{equation}
 \alpha_{\NC}=-\frac{3}{8}\eta ^{-4/5}(2 \eta -1) \Lambda ^2\,,
 \end{equation}
with $a=4$. Similarly, using Eq. \eqref{noncom:gamma_fdot} to Eq. \eqref{eq:2v} gives
 \begin{equation}
 \beta_{\NC}=-\frac{75}{256}\eta ^{-4/5}(2 \eta -1) \Lambda ^2\,, 
 \end{equation}
with $b=-1$. We can see that the noncommutative correction eneters in the phase and amplitude at 2PN order. Comparing  fractional noncommutative deviation of phase \st{I cannot find a better term. Please look at section V.C of \cite{Kobakhidze:2016cqh} } from GR at 2PN order with result computed by LIGO/Virgo collaboration from GW 150914 \cite{TheLIGOScientific:2016src}, constraint on $\Lambda$ was found as $\sqrt{\Lambda} \lesssim 3.5$ \cite{Kobakhidze:2016cqh}, which gives the energy scale of noncommutativity to be the order of Planck scale. This is so far the most stringent constraint on noncommutative scale, which is 15 orders of magnitude stronger compared to the bounds coming from particle physics processes.
 
 \newpage
 \subsection{Varying-G Theory}\label{gdot}
In varying $G$ theories, masses and Newton's constant $G$ are time dependent. If the object has enough gravitational self-energy, the rate at which the mass of the object varies in time is proportional to the rate at which the gravitational coupling constant changes \cite{PhysRevLett.65.953}. Since the formalism of Sec. \ref{section:ppE} requires $G$ and the masses to be constant, we cannot use it for varying-G theories. Rather we will promote binary masses and the Newton's constant to a time dependent form in the following way

 \begin{eqnarray}\label{eq:3.7a2}
 m_1(t)\approx m_{1,0}+\dot{m}_{1,0}(t-t_0)\,, \\
 \label{eq:3.7a3}  m_2(t)\approx m_{2,0}+\dot{m}_{2,0}(t-t_0)\,, \\
   \label{eq:3.7a4}  G(t)\approx  G_0+\dot{G}_0(t-t_0)\, , 
 \end{eqnarray}
where $t_0$ is the time of coalescence. Subscript $0$ denotes the quantity measured at time $t=t_0$. Total mass of the binary varies as
 \begin{equation}
 m(t)=m_0+\dot{m}_0(t-t_0)\,.
 \end{equation}
 GW emission makes the orbital seperation $r$ smaller with the orbital decay rate given by \cite{PhysRevD.49.2658}
 \begin{equation}
 \dot{r}_{GW}=-\frac{64}{5}\frac{G^3 \mu m^2}{r^3}\,,
 \end{equation}
 in $c=1$ unit. On the other hand, time variation of mass and grvatitaional coupling constant changes $r$ at a rate of 
 \begin{equation}
 \dot{r}_H=-\left(\frac{\dot{G}_0}{G}+\frac{\dot{m}_0}{m}\right)\,,
 \end{equation}
 which is derived from the conservation of specific angular momentum $j=\sqrt{Gmr}$. From Kepler's law, evolution of orbital angular frequency is given by
 \begin{equation}\label{eq:3.7a}
 \dot{\Omega}=\frac{1}{2\Omega r^3}\left(m\dot{G}_0+\dot{m}_0G-3mG\frac{\dot{r}}{a}\right)\,.
 \end{equation}
Using binary seperation shift $\dot{r}=\dot{r}_{GW}+\dot{r}_H$ in equation \eqref{eq:3.7a} we can find the GW frequency evolution upto 2PN order as
\begin{align} \label{eq:3.7b}
 \dot{f}=\frac{\dot{\Omega}}{\pi}=& \frac{96}{5}\pi^{8/3}G^{5/3}\mathcal{M}^{5/3}f^{11/3}\left[1+\frac{5}{48 G^{8/3}\eta}(\dot{m}_0G+ \right.\nonumber\\ & \left. m\dot{G}_0)x^{-4} -\left(\frac{743}{336}+\frac{11}{4}\eta\right)x+4\pi x^{3/2}\right.\nonumber\\ & \left. + \left(\frac{34103}{18144}+\frac{13661}{2016}\eta+\frac{59}{18}\eta^2\right)x^2 \right]\,,
\end{align}
where $x=v^2=(\pi M f)^{2/3}$ is the squared velocity of the relative motion. Here we considered only leading order correction to frequency evolution which enters in -4PN order. We can integrate equation \eqref{eq:3.7b} to obtain time before coalescence $t(f)$ and the GW phase $\phi(f)=\int 2 \pi f dt=\int\frac{2\pi f}{\dot{f}}df$ as

\bw
\begin{align}
t(f)=t_0 	-\frac{5}{256}\mathcal{M}_0{G_0}^{-5/3}{u_0}^{-8}\left\{1 \right. & \left. +\left[\frac{5}{512 {m}_0{G_0}^{5/3}{\eta_0}^2}(\dot{m}_{1,0}m_{2,0}+m_{1,0}\dot{m}_{2,0})-\frac{5}{1536{G_0}^{8/3}\eta_0}(11m_0\dot{G}_0+17\dot{m}_0 G_0)\right]{x_0}^{-4}\right.\nonumber\\&\left. +\frac{4}{3}\left(\frac{743}{336}+\frac{11}{4}{\eta_0}\right)x_0 -\frac{32}{5}\pi {x_0}^{3/2}+2\left(\frac{3058673}{1016064}+\frac{5029}{1008}{\eta_0}+\frac{617}{144}{\eta_0}^2\right){x_0}^2 \right\}\,,
\end{align}

and
\begin{align}
\phi(f)=\phi_0  -\frac{1}{16}{G_0}^{-5/3}{u_0}^{-5}\left\{1 \right. & \left. +\left[\frac{25}{3328 m_0 {G_0}^{5/3}{\eta_0}^2}(\dot{m}_{1,0}m_{2,0}+m_{1,0}\dot{m}_{2,0})-\frac{25}{9984{G_0}^{8/3} \eta_0}(11m_0 \dot{G}_0+17\dot{m}_0 G_0)\right]{x_0}^{-4} \right.\nonumber\\& \left. +\frac{5}{3}\left(\frac{743}{336}+\frac{11}{4}{\eta_0}\right){x_0} -10 \pi {x_0}^{3/2} +5\left(\frac{3058673}{1016064}+\frac{5029}{1008}{\eta_0}+\frac{617}{144}{\eta_0}^{2}\right){x_0}^2\right\}\,.
\end{align}


%where the subscript $0$ denotes the quantity measured at time of coalescence.
The GW phase in the Fourier space is given by,

\begin{align}\label{eq:3.7c}
\Psi(f)=&2\pi ft(f)-\phi(f)-\frac{\pi}{4}\nonumber\\
=&2\pi f t_0-\phi_0-\frac{\pi}{4}+\frac{3}{128}{G_0}^{-5/3}{u_0}^{-5}\left\{1+\left[\frac{25}{6656m_0 {G_0}^{5/3}{\eta_0}^2}(\dot{m}_{1,0}m_{2,0}+m_{1,0}\dot{m}_{2,0}) \right.\right.\nonumber\\ &\left.\left. -\frac{25}{19968{G_0}^{8/3}\eta_0}(11m_0\dot{G}_0+17\dot{M}_0G_0)\right]{u_0}^{-8}+\left(\frac{3715}{756} +\frac{55}{9}{\eta_0}\right){x_0}-16\pi {x_0}^{3/2} \right. \nonumber\\ &\left. +\left(\frac{15293365}{508032}+\frac{27145}{504}{\eta_0}+\frac{3085}{72}{\eta_0}^2\right){x_0}^2\right\}\,.
\end{align}

From equation \eqref{eq:3.7c}, $b=-13$  and we can find $\beta$ as
 \begin{align}
 \beta_{\dot{G}}=\frac{-75 \mathcal{M}_0}{851968 {G_0}^{10/3}}\left(\frac{11 \dot{G}_0}{3 G_0}+ \frac{17 \dot{m}_0}{3m_0}- \frac{m_{1,0}\dot{m}_{2,0}+\dot{m}_{1,0}m_{2,0}}{{m_0}^2 \eta0}\right)\,.
  \end{align}
 \ew
In order to calculate $\alpha$ we have to write the amplitude in Fourier space explicitlty in terms of $G$ and the binary mass parameters, which is given by Eq. \eqref{eq:2e}. Further using Kepler's law in Eq. \eqref{eq:2e} gives,
%For a two-body quasi-circular orbit we can write metric perturbation as \cite{Blanchet:2002av}
 %\begin{equation}
%\bar{h}^{ij}(t)\propto \frac{G(t)}{D_L}\frac{d^2 }{d t^2}Q^{ij}\,,
 %\end{equation}
%which gives the amplitude in Fourier space
\begin{align}\label{eq:3.7d}
\tilde{\mathcal{A}}(f)\propto\frac{1}{\sqrt{\dot{f}}}\frac{G(t)}{D_L}\mu(t) r(t)^2f^2\propto\frac{1}{\sqrt{\dot{f}}}{G(t)}^{5/3}\mu(t){m(t)}^{2/3} \,.
\end{align} 
%Here $Q^{ij}$ is the quadruple moment tensor and $D_L$ is the luminosity distance
Using Eqs. \eqref{eq:3.7a2}, \eqref{eq:3.7a3}, and \eqref{eq:3.7a4} in equation \eqref{eq:3.7d} and keeping only leading order correction terms, we can write the amplitude in Fourier space as
\begin{equation}
\tilde{\mathcal{A}}(f)=\tilde{\mathcal{A}}_{\GR}\left(1+\alpha_{\dot{G}}{u_0}^{-8}\right)\,,
\end{equation}
where
\begin{equation}\label{eq:3.7d2}
 \alpha_{\dot{G}}=\frac{-5\mathcal{M}_0}{512 {G_0}^{5/3}} \left(\frac{7 \dot{G}_0}{ G_0} + \frac{5\dot{m}_0}{m_0}+\frac{m_{1,0}\dot{m}_{2,0}+\dot{m}_{1,0}m_{2,0}}{{m_0}^2 \eta0}\right)\,.
 \end{equation}
$\alpha_{\dot{G}}$ in Eq. \eqref{eq:3.7d2} does not match with the previously obtained $\alpha$ for varying-$G$ theory in Ref. \cite{Yunes:2009bv}.
 
 %\subsubsection*{GW Frequency Evolution: Energy-Balance Equation}



%%%%%%%%%%%%%%%%%%%%%%%%%%%%%%%%%%%%%%%%%%%%%%%%%%%%%%%%%%%%%%%%%%%%%%%%

\begin{comment}
  \newpage
 \section{Table}
\begin{tabular}{ |p{1cm}|p{6.9cm}|p{0.4cm}|p{6cm}|p{0.3cm}|}
 \hline
 \multicolumn{5}{|c|}{ppE Parameters}\\
 \hline
 \tiny Theories& $\beta$ & $b$ & $\alpha$& a\\
 \hline
 \vspace{20pt}
   \tiny EdGB &\rule{0pt}{4ex}\tiny$\frac{-5}{7168}\zeta_{EdGB}\frac{(m_1^2s_2^{EdGB}-m_2^2s_1^{EdGB})^2}{m^4\eta^{\frac{18}{5}}}$&\tiny-7& \tiny $\bm{\frac{-5}{192}\zeta_{EdGB}\frac{(m_1^2s_2^{EdGB}-m_2^2s_1^{EdGB})^2}{m^4\eta^{\frac{18}{5}}}}$ &\tiny-2\\  
    \hline
   \vspace{20pt}
\tiny Scalar-Tensor&\rule{0pt}{4ex}\tiny$\frac{-5}{1792}\dot{\phi}^2\eta^{\frac{2}{5}}(m_1s_1^{ST}-m_2s_2^{ST})^2$&\tiny-7&\tiny $\frac{-5}{48}\dot{\phi}^2\eta^{\frac{2}{5}}(m_1s_1^{ST}-m_2s_2^{ST})^2$ &\tiny-2\\
 \hline
  \vspace{20pt}
\tiny dCS& \rule{0pt}{4ex}\tiny$\frac{1549225 \zeta_{dCS} }{11812864 \eta ^{14/5}}(-2 \text{$\delta_m$} \text{$\chi_a$} \text{$\chi_s$}+\left(1-\frac{16068 \eta }{61969}\right) \text{$\chi_a$}^2+\left(1-\frac{231808 \eta }{61969}\right) \text{$\chi_s$}^2)$ &\tiny -1 &\tiny $\bm{\frac{185627 \zeta_{dCS} }{1107456 \eta ^{14/5}}(-2 \text{$\delta_m$} \text{$\chi_a$} \text{$\chi_s$}+\left(1-\frac{53408 \eta }{14279}\right) \text{$\chi_a$}^2+\left(1-\frac{3708 \eta }{14279}\right) \text{$\chi_s$}^2)}$& \tiny 4\\
\hline
 \vspace{20pt}
\tiny KG&\rule{0pt}{4ex}\tiny$\bm{\frac{-5 \sqrt{\alpha^{KG}}\bigg(\frac{(\beta^{KG}-1)(2+\beta^{KG}+3\lambda^{KG})}{(\alpha^{KG}-2)(\beta^{KG}+\lambda^{KG})}\bigg)^{3/2}\eta ^{2/5} (\text{$s_1$}-\text{$s_2$})^2}{3584(\text{$G_N$} (\text{$s_1$}-1) (\text{$s_2$}-1))^{4/3}}}$&\tiny-7 &\tiny$\bm{\frac{112 }{3}\beta_{KG}}$&\tiny-2\\
\hline
 \vspace{20pt}
 \tiny NC&\rule{0pt}{4ex}\tiny${-\frac{75 (2 \eta -1) \Lambda ^2}{256 \eta ^{4/5}}}$&\tiny-1&\tiny$\bm{-\frac{3 (2 \eta -1) \Lambda ^2}{8 \eta ^{4/5}}}$&\tiny4\\
 \hline
  \vspace{20pt}
\tiny Einstein-$\AE$ther&\rule{0pt}{4ex}\tiny$\bm{-\frac{5 \eta ^{2/5} \left(s_1-s_2\right){}^2 \left(\left(c_{14}-2\right) w_0^3-w_1^3\right)}{3584 c_{14} w_0^3 w_1^3 \left(\text{$G_N$} \left(s_1-1\right) \left(s_2-1\right)\right){}^{4/3}}}$&\tiny-7&\tiny$\bm{-\frac{5 \eta ^{2/5} \left(s_1-s_2\right){}^2 \left(\left(c_{14}-2\right) w_0^3-w_1^3\right)}{96 c_{14} w_0^3 w_1^3 \left(\text{$G_N$} \left(s_1-1\right) \left(s_2-1\right)\right){}^{4/3}}}$&\tiny-2\\

 \hline
 \vspace{20pt}
 \tiny Varying-G Theory&\rule{0pt}{4ex}\tiny $\bm{-\frac{75 m_0 {\eta_0}^{3/5}}{851968 {G_0}^{10/3}} \bigg(\frac{11 \dot{G}}{3 G_0} + \frac{17 \dot{M}}{3M_0}-\frac{m_{1,0}\dot{m_2}+m_{2,0}\dot{m_1}}{{m_0}^2 \eta0}\bigg)}$&\tiny-13&\tiny$\bm{\frac{-5 m_0 {\eta_0}^{3/5}}{512 {G_0}^{5/3}} \bigg(\frac{7 \dot{G}}{ G_0} + \frac{5\dot{M}}{m_0}+\frac{m_{1,0}\dot{m_2}+m_{2,0}\dot{m_1}}{{m_0}^2 \eta0}\bigg)}$&\tiny-8\\
\hline
\end{tabular}
\end{comment}

%%%%%%%%%%%%%%%%%%%%%%%%%%%%%%%%%%%%%%%%%%%%%%%%%%%%%%%%%%%%%%%%%%%%%%%%%%%%
\section{Conclusions}
\label{sec:conclusions}

%%%%%%%%%%%%%%%%%%%%%%%%%%%%%%%
\acknowledgments
We would like to thank Nicol\' as Yunes for fruitful discussions.
K.Y. would like to acknowledge networking support by the COST Action GWverse CA16104.

%%%%%%%%%%%%%%%%%%
 \appendix 
 
 \section{Original ppE Formalism}
 \label{appendix}

In this appendix, we review the original ppE formalism. In particular, we will show how the amplitude and phase corrections depend on the conservative and dissipative corrections, where the former are corrections to the effective potential while the latter are those to the GW luminosity. We will mostly follow~\cite{Chatziioannou:2012rf}.

First, let us introduce the conservative corrections. We modify the reduced effective potential as
 \begin{equation}\label{eq:h}
 V_{\text{eff}}=\left(-\frac{m}{r}+\frac{{L}^2_{z}}{2\mu^2r^2}\right)\left[1+A \left(\frac{m}{r}\right)^p\right]\,,
 \end{equation}
where $m$ is the total mass of the binary and $L_{z}$ is the $z$-component of the angular momentum. $A$ and $p$ show the magnitude and exponent of the non-GR correction term respectively. Such a modification to the effective potential also modifies the Kepler's law. Taking the radial derivative of $V_{\text{eff}}$ in Eq.~\eqref{eq:h} and equating it to zero gives the modified Kepler's law as
 \begin{equation}
 \Omega^2=\frac{m}{r^3} \left[1+\frac{1}{2} A \, p\left(\frac{m}{r}\right)^p\right]\,.
 \end{equation}
 The above equation further gives the orbital separation as
 \begin{equation}\label{eq:g}
 r(t)=r_{\GR}\left[1+\frac{1}{6}A\, p\, \eta^{-\frac{2p}{5}}u^{2p}\right]\,,
 \end{equation}
where to leading PN order, $r_{\GR}$ is given by the Kepler's law as $r_{\GR}=(m/\Omega^2)^{1/3}$. For a circular orbit, radial kinetic energy does not exist and the effective potential energy is same as the binding energy of the binary. Using Eq. \eqref{eq:g} in Eq. \eqref{eq:h} and keeping only to leading order in non-GR corrections, the binding energy becomes
\begin{align}\label{eq:j}
E=-\frac{1}{2}\eta^{-2/5}u^2\left[1-\frac{1}{3}A(2p-5)\eta^{-2p/5}u^{2p}\right]\,,
\end{align}
where $u=(\pi \mathcal{M} f)^{1/3}$.

Next, let us introduce dissipative corrections. Such corrections to the GW luminosity can be parameterized by
 \begin{equation}\label{eq:a}
 \dot{E}=\dot{E}_{\GR}\left[1+B\left(\frac{m}{r}\right)^q\right]\,,
 \end{equation}
 where $\dot{E}_{GR}$ is the GR luminosity which is proportional to $v^2(m/r)^4$ with $v=\Omega r = (\pi m f)^{1/3}$ is the relative velocity of binary components\footnote{If we assume $\dot{E}_{\GR}$ to be proportional to $r^4\Omega^6$, we will find slightly different expressions for $\dot{f}$ and the waveform~\cite{Chatziioannou:2012rf}}. 
 
 Let us now derive the amplitude corrections. First, using Eqs.~\eqref{eq:j} and~\eqref{eq:a} and applying the chain rule, the GW frequency evolution is given by
 \begin{align}\label{eq:f}
 \dot{f}&=\frac{df}{dE}\frac{dE}{dt}\nonumber\\ &=\dot{f}_{GR}\left[1+B\eta^{-\frac{2q}{5}} u^{2q}+\frac{1}{3}A(2p^2-2p-3)\eta^{\frac{-2p}{5}}u^{2p} \right]\,,
 \end{align}
where $\dot{f}_{\GR}$ is given by Eq. \eqref{eq:2s}. 
Next, using Eqs.~\eqref{eq:g} and~\eqref{eq:f} to Eq.~\eqref{eq:2e} and keeping only to leading order in non-GR corrections, the GW amplitude in Fourier domain becomes
\begin{align}\label{eq:o2}
\tilde{\mathcal{A}}(f)=\tilde{\mathcal{A}}_{GR} \left[1-\frac{B}{2}\eta^{\frac{-2q}{5}}u^{2q}-\frac{1}{6}A(2p^2-4p-3)\eta^{\frac{-2p}{5}}u^{2p}\right]\,.
\end{align}


Next, we move onto deriving phase corrections. One can derive the GW phase in the Fourier domain by integrating Eq.~\eqref{eq:2r} twice. Equivalently, one can use the following expression
\begin{equation}\label{eq:n}
\Psi(f)=2\pi f t(f)-\phi(f)-\frac{\pi}{4}\,,
\end{equation}
where $t(f)$ gives the relation between time and frequency and can be obtained by integrating~\eqref{eq:f} as
\begin{align}\label{eq:l}
t(f)=&\int \frac{dt}{df} \, df\nonumber\\=&t_0 -\frac{5 \mathcal{M}}{256 u^8}\left[1+\frac{4}{3}A\frac{\left(2 p^2-2 p-3\right)}{(p-4)}\eta ^{-\frac{2 p}{5}} u^{2 p}\right.\nonumber\\ &\left. +\frac{4}{q-4}B\eta ^{-\frac{2q}{5}} u^{2q}\right]\,,
\end{align}
with $t_0$ representing the time of coalescence and keeping only the Newtonian term and leading order non-GR corrections. On the other hand, $\phi(f)$ in Eq.~\eqref{eq:n} corresponds to the GW phase in the time domain and can be calculated from Eq.~\eqref{eq:f} as
\begin{align}\label{eq:m}
\phi(f)=&\int 2 \pi f dt=\int\frac{2\pi f}{\dot{f}}df\nonumber\\
=&\phi_0 -\frac{1}{16 u^5}\left[1+\frac{5}{3}A\frac{\left(2 p^2-2 p-3\right)}{(2 p-5)} \eta ^{-\frac{2 p}{5}} u^{2 p}\right. \nonumber \\
& \left. +\frac{5}{2 q-5}B\eta ^{-\frac{2 q}{5}} u^{2 q}\right]\,,
\end{align}
with $\phi_0$ representing the coalescence phase. Using Eqs.~\eqref{eq:l} and~\eqref{eq:m} into~\eqref{eq:n} and writing $\Psi(f)$ as $\Psi_{\GR}(f)+\delta\Psi(f)$, non-GR modifications to the phase can be found as
 \begin{align}\label{eq:p2}
\delta\Psi(f)=&-\frac{5}{32}A\frac{2p^2-2p-3}{(4-p)(5-2p)}\eta^{\frac{-2p}{5}}u^{2p-5}\nonumber\\ &-\frac{15}{32}B\frac{1}{(4-q)(5-2q)}\eta^{-\frac{2q}{5}}u^{2q-5}\,,
 \end{align}
with $\Psi_{\GR}$ to leading PN order is given by \cite{Blanchet:1995ez}
\begin{equation}
\label{eq:Psi_GR}
\Psi_{\GR}=2\pi f t_0-\phi_0-\frac{\pi}{4}+\frac{3}{128}u^{-5}\,.
\end{equation}

We can easily rewrite the above expressions using $\gamma_r$ and $c_r$. 
Comparing Eq.~\eqref{eq:g} with Eq.~\eqref{eq:2k} we find
\begin{equation}\label{eq:i}
A=\frac{12\gamma_r}{c_r}\eta^{\frac{c_r}{5}}\,, \quad p=\frac{c_r}{2}\,.
\end{equation}
Using this, we can rewrite the GW amplitude in Eq.~\eqref{eq:o2} as
\begin{align}\label{eq:o}
\tilde{\mathcal{A}}(f)=\tilde{\mathcal{A}}_{GR} \left[1-\frac{B}{2}\eta^{\frac{-2q}{5}}u^{2q}-\frac{\gamma_r}{c_r}(c^2_r-4c_r-6)u^{c_r}\right]\,.
\end{align}
Similarly, one can rewrite the correction to the GW phase in Eq.~\eqref{eq:p2} as
\begin{align}\label{eq:p}
\delta\Psi(f)=&-\frac{15}{8}\frac{\gamma_r}{c_r}\frac{c_r^2-2c_r-6}{(8-c_r)(5-c_r)}u^{c_r-5}\nonumber\\ &-\frac{15}{32}B\frac{1}{(4-q)(5-2q)}\eta^{-\frac{2q}{5}}u^{2q-5}\,.
 \end{align}

\section{Varying-G Theory: Frequency Evolution From Energy-Balance Equation}

 
We now show an alternative approach to find $\dot f$ in Eq.~\eqref{eq:3.7b} by applying the energy balance law used in~\cite{Yunes:2009bv}. Total energy of the binary is given by $E=-(G\mu m)/2r$. In order to calculate the leading order correction to the frequency evolution, we can use Kepler's law to rewrite the binding energy as 
 \begin{equation}\label{eq:3.7e}
 E(f,G,m_1,m_2)=-\frac{1}{2}\mu (Gm\Omega)^{2/3}\,,
 \end{equation}
 where $\Omega=\pi f$ is the orbital angular frequency. Using \eqref{eq:3.7a2} - \eqref{eq:3.7a4} in  \eqref{eq:3.7e}, the rate of change of energy becomes
 \begin{align}\label{eq:3.7j}
 \frac{d E}{d t}=\frac{\pi^{2/3}}{6f^{1/3}G^{1/3}m^{4/3}}\left[-3fGm(\dot{m}_1m_2+m_1\dot{m}_2)\right.\nonumber\\ \left.-2m^3\eta(G\dot{f}+f\dot{G})+m^2fG\eta\dot{m}\right]\,.
 \end{align}
 
In GR, such time variation in the binding energy needs to be balanced with the GW luminosity emitted from the system. In varying-$G$ theories, there is an additional contribution due to the variation in $G$ and masses. Namely, the binding energy is not conserved even in the absence of GW emission. To estimate such additional contribution, we rewrite the binding energy in terms of specific angular momentum as
 \begin{equation}\label{eq:3.7f}
 E(G,m_1,m_2,j)=-\frac{G^2 \mu  m^2}{2 j^2}\,.
 \end{equation}
Taking the time variation of the above binding energy and adding the GW luminosity, the energy-balance equation for varying-$G$ theories then becomes
 \begin{equation}\label{eq:3.7g}
\frac{d E}{d t}=-\dot{E}_{GW}+\frac{\partial E}{\partial m_1}\dot{m_1}+\frac{\partial E}{\partial m_2}\dot{m_2}+\frac{\partial E}{\partial G}\dot{G}\,,
 \end{equation}
 where $E$ is given by equation \eqref{eq:3.7f} and $\dot{E}_{GW}$ is the energy radiated by GWs. Using the quadruple formula, the first term in the above equation is given by
 \begin{equation}\label{eq:3.7h}
 \dot{E}_{GW}=\frac{1}{5}\left \langle\dddot{Q}_{ij}\dddot{Q}_{ij}-\frac{1}{3}(\dddot{Q}_{kk})^2\right \rangle=\frac{32}{5} r^4 G \mu ^2 \Omega ^6\,.
 \end{equation}
 The last term in Eq.~\eqref{eq:3.7g} was missing in~\cite{Yunes:2009bv}.
 %
%\hspace{15.5pt}
 Using \eqref{eq:3.7f} and \eqref{eq:3.7h} in \eqref{eq:3.7g},
 \begin{align}\label{eq:3.7i}
\frac{d E}{d t}=- \frac{32}{5} \pi ^{10/3} f^{10/3} \eta ^2 G^{7/3} m^{10/3}-\frac{G^2}{2j^2}(m_1^2\dot{m}_2+\dot{m}_1 m_2^2)\nonumber\\-\frac{Gm^2\eta}{j^2}(m\dot{G}+\dot{m}G)\,.
 \end{align}
Solving equation \eqref{eq:3.7i} and \eqref{eq:3.7j} we can find frequency evolution upto -4PN order as
 
 \begin{align} 
 \dot{f}=\frac{96}{5}\pi^{8/3}G^{5/3}\mathcal{M}^{5/3}f^{11/3}[1+\frac{5\eta^{3/5}}{48 G^{8/3}}(\dot{m}G+m\dot{G})u^{-8}]\,,
 \end{align} 
 
 
 where $u=(\pi \mathcal{M}f)^{1/3}$. Including higher order GR corrections we can write frequency evolution upto 2PN order, which matches with Eq.~\eqref{eq:3.7b}.
 
 %\begin{align}
% \dot{f}=\frac{96}{5}\pi^{8/3}G^{5/3}\mathcal{M}^{5/3}f^{11/3}\bigg[1+\frac{5\eta^{3/5}}{48 G^{8/3}}(\dot{M}G+M\dot{G})u^{-8}-(\frac{743}{336\eta^{2/5}}+\frac{11}{4}\eta^{3/5})u^2\\+4\pi\eta^{-3/5}u^3+(\frac{34103}{18144\eta^{4/5}}+\frac{13661}{2016}\eta^{1/5}+\frac{59}{18}\eta^{6/5})u^4\bigg]
% \end{align}

%\include{reference}
%\bibliographystyle{plain}
\bibliography{bibfile}
\end{document}