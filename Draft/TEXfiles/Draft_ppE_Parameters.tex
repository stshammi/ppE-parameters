\documentclass[11pt]{article}
\usepackage{amsmath}
\usepackage{amssymb}
\usepackage{amsthm}
\usepackage{cite}
\usepackage{hyperref}
\usepackage{graphicx}
\graphicspath{ {images/} }
%\hypersetup{colorlinks=true, urlcolor=magenta,linkcolor=blue,citecolor=green}
\usepackage{tensor}
\usepackage{mathrsfs}
\usepackage{romanbar}
\usepackage{tabularx}
\usepackage{appendix}
\usepackage{comment} 
\usepackage{bm}
\usepackage{titlesec}
\begin{document}
\title{post-Einsteinian Parameters for Different Modified Gravity Theories}
\maketitle
\section{Introduction}

\section{ppE Paramters}\label{section:intro}
\hspace*{15.5pt}Fourier waveform for inspiral phase in standard ppE framework \cite{Yunes:2009ke},
\begin{equation}\label{eq:1}
\tilde{h}(f)=\tilde{h}_{GR}(1+\alpha u^a)e^{\delta\Psi}
\end{equation}
\begin{equation}\label{eq:2}
\delta\Psi=i\beta u^b
\end{equation}
\begin{equation}
u=(\mathcal{M}_c \Omega)^\frac{1}{3}
\end{equation}
\hspace*{20pt}In above equations $\alpha$, $\beta$, $a$, and $b$ are ppE parameters. $\Omega$ is the orbital angular frequency and $\mathcal{M}_c$ is the chirp mass. In order to obtain the expression of these parmeteres for different modified gravity theories we consider correction to binding energy $E$ and rate of change of binding energy $\dot{E}$. These corrections lead to the modification of Kepler's law and frequency evolution $\dot{f}$.\\
\begin{comment}
\subsection{Disspative Correction Only}
\hspace{15.5pt}First we want to  derive ppE parameters considering only correction to the rate of change of binding energy i.e. correction to frequency evolution. We want to write $\dot{f}$ as
\begin{equation}\label{5}
\dot{f}=\dot{f}_{GR}(1+\gamma_{\dot{f}} u^{c_{\dot{f}}})
\end{equation}
where $\gamma_{\dot{f}}$ and $c_{\dot{f}}$ are ppE parameters. We want to relate $\gamma_{\dot{f}}$ and $c_{\dot{f}}$ to $\alpha$ and $a$. We also want to find a relation between $\alpha$ and $\beta$.\\
\hspace*{15.5pt}Gravitational waveform phase for the dominant harmonic in the Fourier domain satisfies the relation \cite{Tichy:1999pv}\cite{Stein:2013wza}
\begin{equation}\label{3}
\frac{d^2\Psi}{d\Omega^2}=2\frac{dt}{d\Omega}
\end{equation}
By using modfication in ppE phase from \eqref{eq:2} in left side of equation \eqref{3},
\begin{equation}\label{7}
\frac{d^2\delta\Psi}{d\Omega^2}=\frac{\beta b}{3}(\frac{b}{3}-1){\mathcal{M}_c}^2u^{b-6}
\end{equation}
In GR leading order term in orbital frequency evolution is given by \cite{Aubert:2006sb},
\begin{equation}\label{4}
\dot{F}_{GR}=\frac{48}{5\pi {\mathcal{M}_c}^2}u^{11}
\end{equation}
where $F=\frac{f}{2}$.\\
Using equation \eqref{4} and \eqref{5} in the right side of equation \eqref{3},
\begin{equation}\label{6}
2\frac{dt}{d\Omega}=\frac{5\pi}{24}{\mathcal{M}_c}^2u^{-11}-\gamma\frac{5\pi}{24}{\mathcal{M}_c}^2u^{-11+c}
\end{equation}
Comparing equation \eqref{6} and \eqref{7} we find the relation among ppE parameters as following
\begin{equation}
\beta=\frac{15}{8}\frac{\alpha}{b(b-3)}
\end{equation}
\begin{equation}
\alpha=-\frac{1}{2}\gamma_{\dot{f}}
\end{equation}
\begin{equation}
a=c_{\dot{f}}
\end{equation}
\begin{equation}
b=c_{\dot{f}}-5
\end{equation}
\begin{equation}
\beta(\gamma_{\dot{f}},c_{\dot{f}})=\frac{15}{16}\frac{\gamma_{\dot{f}}}{(c_{\dot{f}}-5)(c_{\dot{f}}-8)}
\end{equation}
\end{comment}
 %\subsection{Dissipative and Conservative Correction}
\hspace*{15.5pt}For conservative correction we need to consider modification to binding energy which can be parametrized as follows\cite{Chatziioannou:2012rf}
 \begin{equation}
 E=E_{GR}[1+A(\frac{m}{r})^p]
 \end{equation}
 where we assume A is small, such that the correction represents a small deformation away from GR. Such a binding energy modifies Kepler’s third law as
 \begin{equation}
 \Omega^2=\frac{m}{r^3} [1+\frac{1}{2} Ap(\frac{m}{r})^p]
 \end{equation}
  Above equation gives us orbital seperation as
 \begin{equation}\label{eq:g}
 a(t)=a_{GR}[1+\frac{1}{6}Apu^{2p}\eta^{-\frac{2p}{5}}]
 \end{equation}
 
 \hspace*{15.5pt}Dissipative correction to waveform comes from the modification of rate of change of binding energy. Let us assume a modification to the rate of change of binding energy of the form\cite{Chatziioannou:2012rf}
 
 \begin{equation}\label{eq:a}
 \dot{E}=\dot{E}_{GR}[1+B(\frac{m}{r})^q]
 \end{equation}
 where $\dot{E}_{GR}$ is the GR energy flux and the second term is assumed small relative to the first, as we are interested in small deformations away from GR. By invoking energy balance law equation \eqref{eq:a} gives us the frequency evolution as 
 \begin{equation}\label{eq:f}
 \dot{f}=\dot{f}_{GR}[1+B\eta^\frac{-2q}{5} u^{2q}+\frac{1}{3}A(2p^2-2p-3)\eta^{\frac{-2p}{5}}u^{2p}]
 \end{equation}\label{eq:e}
 \hspace{15.5pt}From stationary phase approximation\cite{Yunes:2009yz},
 \begin{equation}\label{eq:b}
 \tilde{h}(f)=\frac{\mathcal{A}(t_0)}{\sqrt{l\dot{F}}}e^{-i\Psi}
 \end{equation}
 \subsubsection*{Correction to the Amplitude}
\hspace*{15.5pt}We consider correction to the quadruple radiation only. So, we will take $l=2$ in \eqref{eq:b} which gives the amplitude of Fourier waveform as
\begin{equation}\label{eq:c}
\tilde{\mathcal{A}}(f)=\frac{\mathcal{A}(t_0)}{\sqrt{\dot{f}}}
\end{equation}
$\dot{f}$ is the frequency evolution of gravitational wave. $\mathcal{A}(t_0)\sim\ddot{Q}$ where $Q$ is the quadruple moment which is proportional to ${a(t_0)}^2$. So we can write

\begin{equation}\label{eq:d}
\mathcal{A}(f)\sim \frac{a(t)^2}{\sqrt{\dot{f}}}
\end{equation}
Using equation \eqref{eq:f} and \eqref{eq:g} in \eqref{eq:d} and keeping only leading order terms,

\begin{equation}
\tilde{\mathcal{A}}(f)=\tilde{\mathcal{A}}_{GR} [1-\frac{B}{2}\eta^{\frac{-2q}{5}}u^{2q}-\frac{1}{6}A(2p^2-2p-3)\eta^{\frac{-2p}{5}}u^{2p}+\frac{1}{3}Ap\eta^{\frac{-2p}{5}}u^{2p}]
\end{equation}
When dissipative correction dominates $(q>p)$, comparing above equation with the ppE waveform in \eqref{eq:1},
\begin{equation}
\alpha=-\frac{1}{2}B\eta^{\frac{-2q}{5}},\hspace*{20pt}a=2q
\end{equation}
When conservative correction dominates $(q<p)$,
\begin{equation}
\alpha=\frac{1}{6}A(4p-2p^2+3)\eta^{\frac{-2p}{5}},\hspace*{20pt}a=2p
\end{equation}
If the modifications to the binding energy enter at the same PN order as the modifications to the energy flux, ie. $q=k=p$, then
\begin{equation}
\alpha=(\frac{1}{6}A(4k-2k^2+3)-\frac{B}{2})\eta^{\frac{-2k}{5}},\hspace*{20pt}a=2k
\end{equation}
 \subsubsection*{Correction to the Phase}
 
 \hspace*{15.5pt}Modification to the fourier phase in leading order can be calculated as \cite{Chatziioannou:2012rf}
 \begin{equation}
 \delta\Psi=-\frac{5}{32}A\frac{2p^2-2p-3}{(4-p)(5-2p)}\eta^{\frac{-2p}{5}}u^{2p-5}-\frac{15}{32}B\frac{1}{(4-q)(5-2q)}\eta^{\frac{-2q}{5}}u^{2q-5}
 \end{equation}
 Comparing with \eqref{eq:1} we find ppE parameters $\beta$ and $b$ as
 \begin{equation}
 q>p,\hspace*{20pt}\beta=-\frac{15}{32}B\frac{1}{(4-q)(5-2q)}\eta^{\frac{-2q}{5}},\hspace*{20pt},b=2q-5
 \end{equation}
 \begin{equation}
 q<p,\hspace*{20pt}\beta=-\frac{5}{32}A\frac{2p^2-2p-3}{(4-p)(5-2p)}\eta^{\frac{-2p}{5}},\hspace*{20pt}b=2p-5
 \end{equation}
 \begin{equation}
 q=p=k,\hspace*{20pt}\beta=-(\frac{5}{32}A\frac{2k^2-2k-3}{(4-k)(5-2k)}+\frac{15}{32}B\frac{1}{(4-k)(5-2k)})\eta^{\frac{-2k}{5}}
 \end{equation}
 \section*{Alternative}
 \hspace*{15.5pt}We want to write orbital seperation as,
 \begin{equation}
 a=a_{GR}(1+\gamma_au^{c_a})
 \end{equation}
and frequency evolution as,
\begin{equation}
\dot{f}=\dot{f}_{GR}(1+\gamma_{\dot{f}}u^{c_{\dot{f}}})
\end{equation} 
Comparing with \eqref{eq:g},
\begin{equation}
c_{a}=2p \text{,\hspace{20pt}}\gamma_a=\frac{1}{6}Ap\eta^{-2p/5}
\end{equation}
\hspace*{15.5pt}$\gamma_{\dot{f}}$ and $c_{\dot{f}}$ can take different forms depending on whether the dominative correction is dissipative or conservative.
\subsubsection*{Dissipative $(p<q)$}
\hspace*{15.5pt}Comparing with \eqref{eq:f},
\begin{equation}
\gamma_{\dot{f}}=B\eta^\frac{-2q}{5}\text{,\hspace*{20pt}}c_{\dot{f}}=2q
\end{equation}
ppE parameters,
\begin{equation}
\beta=-\frac{15 \text{$\gamma_{\dot{f}} $}}{16 (\text{$c_{\dot{f}}$}-8) (\text{$c_{\dot{f}}$}-5)}
\end{equation}
\begin{equation}
b=c_{\dot{f}}-5\text{,\hspace*{20pt}}a=c_{\dot{f}}
\end{equation}
\begin{equation}
\alpha=-\frac{\gamma_{\dot{f}}}{2}
\end{equation}
\subsubsection*{Conservative$(p>q)$}
\begin{equation}
\gamma_{\dot{f}}=\frac{1}{3}A(2p^2-2p-3)\eta^{\frac{-2p}{5}}\text{,\hspace*{20pt}}c_{\dot{f}}=c_a=2p
\end{equation}
ppE parameters,
\begin{equation}
\beta=-\frac{15 \gamma_{\dot{f}} }{16 (c_{\dot{f}}-8) (c_{\dot{f}}-5)}
\end{equation}
\begin{equation}
b=c_{\dot{f}}-5\text{,\hspace*{20pt}}a=c_a
\end{equation}
\begin{equation}
\alpha=-\frac{\gamma_a}{c_a}(c_a^2-4 c_a-6)
\end{equation}
\subsubsection*{Dissipative=Conservative $(p=q=k)$}
\begin{equation}
\gamma_{\dot{f}}=B\eta^\frac{-2k}{5}+\frac{1}{3}A(2k^2-2k-3)\eta^{\frac{-2k}{5}}\text{,\hspace*{20pt}}c_{\dot{f}}=2k
\end{equation}
ppE parameters,
\begin{equation}
\beta=-\frac{15 \text{$\gamma_{\dot{f}} $}}{16 (\text{$c_{\dot{f}}$}-8) (\text{$c_{\dot{f}}$}-5)}
\end{equation}
\begin{equation}
b=c_{\dot{f}}-5\text{,\hspace*{20pt}}a=c_{\dot{f}}
\end{equation}
\begin{equation}
\alpha=2 \text{$\gamma_a $}-\frac{\text{$\gamma_{\dot{f}} $}}{2}
\end{equation}
\vspace{50pt}
 \section{Example Theories}
 \vspace*{20pt}
 \subsection{EdGB Theory}
 Dissipative correction dominates. From \cite{Yunes:2016jcc} and \cite{Yagi:2011xp},
 
 \begin{equation}
 \beta_{EdGB}=\frac{-5}{7168}\zeta_{EdGB}\frac{(m_1^2s_2^{EdGB}-m_2^2s_1^{EdGB})^2}{m^4\eta^{\frac{18}{5}}}
 \end{equation}
 \begin{equation}
 \alpha_{EdGB}=\frac{-5}{192}\zeta_{EdGB}\frac{(m_1^2s_2^{EdGB}-m_2^2s_1^{EdGB})^2}{m^4\eta^{\frac{18}{5}}}
 \end{equation}
 \subsection{Scalar-Tensor Theories}
 Dissipative correction dominates.  From \cite{Yunes:2016jcc},
 \begin{equation}
 \beta_{SC}=\frac{-5}{1792}\dot{\phi}^2\eta^{\frac{2}{5}}(m_1s_1^{ST}-m_2s_2^{ST})^2
 \end{equation}
 \begin{equation}
 \alpha_{ST}fairy tail 2014 episode 55=\frac{-5}{48}\dot{\phi}^2\eta^{\frac{2}{5}}(m_1s_1^{ST}-m_2s_2^{ST})^2
 \end{equation}
 \subsection{DCS Theory}
 Dissipative and conservative correction enters at same order.
From \cite{Yagi:2012vf},
 \begin{equation}
 \beta_{DCS}=\frac{1549225 \zeta_{DCS} }{11812864 \eta ^{14/5}}(-2 \text{$\delta_m$} \text{$\chi_a$} \text{$\chi_s$}+\left(1-\frac{16068 \eta }{61969}\right) \text{$\chi_a$}^2+\left(1-\frac{231808 \eta }{61969}\right) \text{$\chi_s$}^2)
 \end{equation}
 \begin{equation}
 \alpha_{DCS}=\frac{185627 \zeta_{DCS} }{1107456 \eta ^{14/5}}(-2 \text{$\delta_m$} \text{$\chi_a$} \text{$\chi_s$}+\left(1-\frac{53408 \eta }{14279}\right) \text{$\chi_a$}^2+\left(1-\frac{3708 \eta }{14279}\right) \text{$\chi_s$}^2)
 \end{equation}
 \subsection{Einstein-$\AE$ther Theory}
 Dissipative correction dominates. From \cite{Hansen:2014ewa},
 \begin{equation}
 \beta_{\AE}=-\frac{5 \eta ^{2/5} \left(s_1-s_2\right){}^2 \left(\left(c_{14}-2\right) w_0^3-w_1^3\right)}{3584 c_{14} w_0^3 w_1^3 \left(\text{$G_N$} \left(s_1-1\right) \left(s_2-1\right)\right){}^{4/3}}
 \end{equation}
 \begin{equation}
 \alpha_{\AE}=-\frac{5 \eta ^{2/5} \left(s_1-s_2\right){}^2 \left(\left(c_{14}-2\right) w_0^3-w_1^3\right)}{96 c_{14} w_0^3 w_1^3 \left(\text{$G_N$} \left(s_1-1\right) \left(s_2-1\right)\right){}^{4/3}}
 \end{equation}
 
 \subsection{KG Theory}
 Dissipative correction dominates. From \cite{Hansen:2014ewa},
 \begin{equation}
 \alpha_{KG}=-\frac{5 \sqrt{\alpha^{KG}}\bigg(\frac{(\beta^{KG}-1)(2+\beta^{KG}+3\lambda^{KG})}{(\alpha^{KG}-2)(\beta^{KG}+\lambda^{KG})}\bigg)^{3/2}\eta ^{2/5} (\text{$s_1$}-\text{$s_2$})^2}{96 (\text{$G_N$} (\text{$s_1$}-1) (\text{$s_2$}-1))^{4/3}}
 \end{equation}
 \begin{equation}
 \beta_{KG}=-\frac{5 \sqrt{\alpha^{KG}}\bigg(\frac{(\beta^{KG}-1)(2+\beta^{KG}+3\lambda^{KG})}{(\alpha^{KG}-2)(\beta^{KG}+\lambda^{KG})}\bigg)^{3/2}\eta ^{2/5} (\text{$s_1$}-\text{$s_2$})^2}{3584(\text{$G_N$} (\text{$s_1$}-1) (\text{$s_2$}-1))^{4/3}}
 \end{equation}

 \subsection{Non-Commutative Gravity}
 Dissipative and conservative correction enters at same order. \cite{Kobakhidze:2016cqh}
 \begin{equation}
 \alpha_{NC}=-\frac{3 (2 \eta -1) \Lambda ^2}{8 \eta ^{4/5}}
 \end{equation}
 \begin{equation}
 \beta_{NC}=-\frac{75 (2 \eta -1) \Lambda ^2}{256 \eta ^{4/5}}
 \end{equation}
 \newpage
 
 \subsection{Varying-G Theory}

 \hspace*{15.5pt} In varying $G$ theories, masses and Newton's constant $G$ are time dependent. The mass of the bodies with appreciable gravitational self-energy vary in time at a rate proportional to any time variation of gravitational coupling constant\cite{PhysRevLett.65.953}. Since the formalisom of section \ref{section:intro} requires $G$ and the masses to be constant, we cannot use it for varying-G theories. Rather we will promote binary masses and the Newton's constant to a time dependent form in the following way
 
 \begin{eqnarray}\label{eq:3.7a2}
 M_1(t)=M_{1,0}+\dot{M_1}(t-t_0)\\
 \label{eq:3.7a3}  M_2(t)=M_{2,0}+\dot{M_2}(t-t_0)\\
   \label{eq:3.7a4}  G(t)=G_0+\dot{G}(t-t_0)
 \end{eqnarray}
 where $t_0$ is the time of coalescence and $M_1(t)$ and $M_2(t)$ are the masses of the binary components at time $t$. $G_0$, $M_{1,0}$, and $M_{2,0}$ are evaluated at time $t=t_0$. Total mass of the binary varies as
 \begin{equation}
 M(t)=M_0+\dot{M}(t-t_0)
 \end{equation}
 
 \hspace*{15.5pt}GW emission makes the orbital seperation $a$ smaller with the orbital decay rate given by \cite{PhysRevD.49.2658},
 \begin{equation}
 \dot{a}_{GW}=-\frac{64}{5}\frac{G^3 \mu M^2}{a^3}
 \end{equation}
 in $c=1$ unit. On the other hand, time variation of mass and grvatitaional coupling constant makes $a$ larger at a rate of 
 \begin{equation}
 \dot{a}_H=-(\frac{\dot{G}}{G}+\frac{\dot{M}}{M})
 \end{equation}
 which is derived from the conservation of specific angular momentum $j=\sqrt{GMa}$. From Kepler's law, evolution of orbital angular frequency,
 \begin{equation}\label{eq:3.7a}
 \dot{\Omega}=\frac{1}{2\Omega a^3}(M\dot{G}+\dot{M}G-3MG\frac{\dot{a}}{a})
 \end{equation}
 \hspace*{15.5pt} Using binary shift $\dot{a}=\dot{a}_{GW}+\dot{a}_H$ in equation \eqref{eq:3.7a} we can find the GW frequency shift upto 2PN order as
 \begin{align} \label{eq:3.7b}
 \dot{f}=\frac{96}{5}\pi^{8/3}G^{5/3}\mathcal{M}^{5/3}f^{11/3}\bigg[1+\frac{5\eta^{3/5}}{48 G^{8/3}}(\dot{M}G+M\dot{G})u^{-8}-(\frac{743}{336\eta^{2/5}}+\frac{11}{4}\eta^{3/5})u^2\\+4\pi\eta^{-3/5}u^3+(\frac{34103}{18144\eta^{4/5}}+\frac{13661}{2016}\eta^{1/5}+\frac{59}{18}\eta^{6/5})u^4\bigg]
 \end{align}
 
where $u=(\pi \mathcal{M}f)^{1/3}$. Here we considered only leading order correction to frequency evolution which enters in -4PN order. We can integrate equation \eqref{eq:3.7b} to obtain time before coalescence $t(f)$ and the GW phase $\phi(f)$ as


\begin{align}
t(f)=t_0-\frac{5}{256}\mathcal{M}_0{G_0}^{-5/3}{u_0}^{-8}\bigg[1+\big(\frac{5{\eta_0}^{1/5}}{512\mathcal{M}_0{G_0}^{5/3}}(\dot{M_1}M_{2,0}+M_{1,0}\dot{M_{2}})\nonumber\\-\frac{5{\eta_0}^{3/5}}{1536{G_0}^{8/3}}(11M_0\dot{G}+17\dot{M}G_0)\big){u_0}^{-8}+\frac{4}{3}(\frac{743}{336{\eta_0}^{2/5}}+\frac{11}{4}{\eta_0}^{3/5}){u_0}^2\nonumber\\-\frac{32}{5}{\eta_0}^{-3/5}{u_0}^3+2(\frac{3058673}{1016064{\eta_0}^{4/5}}+\frac{5029}{1008}{\eta_0}^{1/5}+\frac{617}{144}{\eta_0}^{6/5}){u_0}^4\bigg]
\end{align}
and
\begin{align}
\phi(f)=\phi_0-\frac{1}{16}{G_0}^{-5/3}{u_0}^{-5}\bigg[1+\big(\frac{25{\eta_0}^{1/5}}{3328{G_0}^{5/3}{\mathcal{M}}_0}(\dot{M_1}M_{2,0}+\dot{M_2}M_{1,0})\nonumber\\-\frac{25{\eta_0}^{3/5}}{9984{G_0}^{8/3}}(11\dot{G}M_0+17\dot{M}G_0)\big){u_0}^{-8}+\frac{5}{3}(\frac{743}{336{\eta_0}^{2/5}}+\frac{11}{4}{\eta_0}^{3/5}){u_0}^2\nonumber\\-10{\eta_0}^{-3/5}{u_0}^3+5(\frac{3058673}{1016064{\eta_0}^{4/5}}+\frac{5029}{1008}{\eta_0}^{1/5}+\frac{617}{144}{\eta_0}^{6/5}){u_0}^4\bigg]
\end{align}

where the subscript $0$ denotes the quantity measured at time of coalescence. The GW phase in the fourier space is given by,
\begin{align}\label{eq:3.7c}
\psi(f)=&2\pi ft(f)-\phi(f)-\frac{\pi}{4}\nonumber\\
=&2\pi f t_0-\phi_0-\frac{\pi}{4}+\frac{3}{128}{G_0}^{-5/3}{u_0}^{-5}\bigg[1+\big(\frac{25{\eta_0}^{1/5}}{6656{G_0}^{5/3}{\mathcal{M}}_0}(\dot{M_1}M_{2,0}+\dot{M_2}M_{1,0})\nonumber\\&-\frac{25{\eta_0}^{3/5}}{19968{G_0}^{8/3}}(11M_0\dot{G}+17\dot{M}G_0)\big){u_0}^{-8}+(\frac{3715}{756{\eta_0}^{2/5}}+\frac{55}{9}{\eta_0}^{3/5}){u_0}^2-16\pi{\eta_0}^{-3/5}{u_0}^3 &\nonumber\\&\qquad\qquad+(\frac{15293365}{508032{\eta_0}^{4/5}}+\frac{27145}{504}{\eta_0}^{1/5}+\frac{3085}{72}{\eta_0}^{6/5}){u_0}^4\bigg]
\end{align}
\hspace*{15.5pt}From equation \eqref{eq:3.7c}, $b=-13$  and we can find $\beta$ as

 \begin{equation}
 \beta_{\dot{G}}=\frac{-75 \mathcal{M}_0}{851968 {G_0}^{10/3}} \bigg(\frac{11 \dot{G}}{3 G_0} + \frac{17 \dot{M}}{3M_0}-\frac{M_{1,0}\dot{M_2}+M_{2,0}\dot{M_1}}{{M_0}^2 \eta0}\bigg)
  \end{equation}
 
 \hspace*{15.5pt}In order to calculate $\alpha$ we have to write metric perturbation explicitlty in terms of $G$ and binary mass parameters. For a two-body quasi-circular orbit we can write metric perturbation as \cite{Blanchet:2002av}
 \begin{equation}
\bar{h}^{ij}(t)\propto \frac{G(t)}{D_L}\frac{\mathrm{d^2} }{\mathrm{d} t^2}Q^{ij}
 \end{equation}
which gives the amplitude in fourier space
\begin{align}\label{eq:3.7d}
\tilde{\mathcal{A}}(f)\propto\frac{1}{\sqrt{\dot{f}}}\frac{G(t)}{D_L}\mu(t) a(t)^2\omega^2\nonumber\\\propto\frac{1}{\sqrt{\dot{f}}}{G(t)}^{5/3}\mu(t){M(t)}^{2/3}
\end{align} 
 

\hspace*{15.5pt}Here $Q^{ij}$ is the quadruple moment tensor and $D_L$ is the luminosity distance. Using equations \eqref{eq:3.7a2}, \eqref{eq:3.7a3}, and \eqref{eq:3.7a4} in equation \eqref{eq:3.7d} and keeping only leading order correction terms, we can write the amplitude in fourier space as
\begin{equation}
\tilde{\mathcal{A}}(f)=\tilde{\mathcal{A}}_{GR}\big(1+\alpha_{\dot{G}}{u_0}^{-8}\big)
\end{equation}
where
\begin{equation}
 \alpha_{\dot{G}}=\frac{-5\mathcal{M}_0}{512 {G_0}^{5/3}} \bigg(\frac{7 \dot{G}}{ G_0} + \frac{5\dot{M}}{M_0}+\frac{M_{1,0}\dot{M_2}+M_{2,0}\dot{M_1}}{{M_0}^2 \eta0}\bigg)
 \end{equation}
 
 \subsubsection*{GW Frequency Evolution: Energy-Balance Equation}
 
 \hspace{15.5pt}Total energy of the binary is given by $E=-\frac{G\mu M}{2a}$. In order to calculate leading order correction to frequency evolution, we can use Kepler's law to rewrite the binding energy as
 \begin{equation}\label{eq:3.7e}
 E(f,G,M_1,M_2)=-\frac{1}{2}\mu (GM\omega)^{2/3}
 \end{equation}
 where $\omega=\pi f$ is the orbital angular frequency. Using \eqref{eq:3.7a2} - \eqref{eq:3.7a4} in  \eqref{eq:3.7e}, the rate of change of energy
 \begin{align}\label{eq:3.7j}
 \frac{\mathrm{d} E}{\mathrm{d} t}=\frac{\pi^{2/3}}{6f^{1/3}G^{1/3}M^{4/3}}\bigg[-3fGM(\dot{M}_1M_2+M_1\dot{M}_2)-2M^3\eta(G\dot{f}+f\dot{G})+M^2fG\eta\dot{M}\bigg]
 \end{align}
 \hspace{15.5pt}Energy can also be expressed in terms of specific angular momentum as
 \begin{equation}\label{eq:3.7f}
 E(G,M_1,M_2,j)=-\frac{G^2 \mu  M^2}{2 j^2}
 \end{equation}
 \hspace*{15.5pt} Energy-balance equation for varying-G theories,
 \begin{equation}\label{eq:3.7g}
\frac{\mathrm{d} E}{\mathrm{d} t}=-\dot{E}_{GW}+\frac{\partial E}{\partial M_1}\dot{M1}+\frac{\partial E}{\partial M_2}\dot{M_2}+\frac{\partial E}{\partial G}\dot{G}
 \end{equation}
 where $E$ is given by equation \eqref{eq:3.7f} and $\dot{E}_{GW}$ is the energy radiated by gravitational wave. Using quadruple formula,
 \begin{equation}\label{eq:3.7h}
 \dot{E}_{GW}=\frac{1}{5}\left \langle\dddot{Q}_{ij}\dddot{Q}_{ij}-\frac{1}{3}(\dddot{Q}_{kk})^2\right \rangle=\frac{32}{5} a^4 G \mu ^2 \omega ^6
 \end{equation}
 \hspace{15.5pt}Using \eqref{eq:3.7f} and \eqref{eq:3.7h} in \eqref{eq:3.7g},
 \begin{align}\label{eq:3.7i}
\frac{\mathrm{d} E}{\mathrm{d} t}=- \frac{32}{5} \pi ^{10/3} f^{10/3} \eta ^2 G^{7/3} M^{10/3}-\frac{G^2}{2j^2}({M_1}^2\dot{M}_2+\dot{M}_1{M_2}^2)-\frac{GM^2\eta}{j^2}(M\dot{G}+\dot{M}G)
 \end{align}
 \hspace{15.5pt}Solving equation \eqref{eq:3.7i} and \eqref{eq:3.7j} we can find frequency evolution upto -4PN order as
 
 \begin{align} 
 \dot{f}=\frac{96}{5}\pi^{8/3}G^{5/3}\mathcal{M}^{5/3}f^{11/3}[1+\frac{5\eta^{3/5}}{48 G^{8/3}}(\dot{M}G+M\dot{G})u^{-8}]
 \end{align} 
 
 where $u=(\pi \mathcal{M}f)^{1/3}$. Including higher order GR corrections we can write frequency evolution upto 2PN order as
 
 
 
 \begin{align}
 \dot{f}=\frac{96}{5}\pi^{8/3}G^{5/3}\mathcal{M}^{5/3}f^{11/3}\bigg[1+\frac{5\eta^{3/5}}{48 G^{8/3}}(\dot{M}G+M\dot{G})u^{-8}-(\frac{743}{336\eta^{2/5}}+\frac{11}{4}\eta^{3/5})u^2\\+4\pi\eta^{-3/5}u^3+(\frac{34103}{18144\eta^{4/5}}+\frac{13661}{2016}\eta^{1/5}+\frac{59}{18}\eta^{6/5})u^4\bigg]
 \end{align}
 
 
 
 \newpage
 \section{Table}

\begin{tabular}{ |p{1cm}|p{6.9cm}|p{0.4cm}|p{6cm}|p{0.3cm}|}
 \hline
 \multicolumn{5}{|c|}{ppE Parameters}\\
 \hline
 \tiny Theories& $\beta$ & $b$ & $\alpha$& a\\
 \hline
 \vspace{20pt}
   \tiny EdGB &\rule{0pt}{4ex}\tiny$\frac{-5}{7168}\zeta_{EdGB}\frac{(m_1^2s_2^{EdGB}-m_2^2s_1^{EdGB})^2}{m^4\eta^{\frac{18}{5}}}$&\tiny-7& \tiny $\bm{\frac{-5}{192}\zeta_{EdGB}\frac{(m_1^2s_2^{EdGB}-m_2^2s_1^{EdGB})^2}{m^4\eta^{\frac{18}{5}}}}$ &\tiny-2\\  
    \hline
   \vspace{20pt}
\tiny Scalar-Tensor&\rule{0pt}{4ex}\tiny$\frac{-5}{1792}\dot{\phi}^2\eta^{\frac{2}{5}}(m_1s_1^{ST}-m_2s_2^{ST})^2$&\tiny-7&\tiny $\frac{-5}{48}\dot{\phi}^2\eta^{\frac{2}{5}}(m_1s_1^{ST}-m_2s_2^{ST})^2$ &\tiny-2\\
 \hline
  \vspace{20pt}
\tiny DCS& \rule{0pt}{4ex}\tiny$\frac{1549225 \zeta_{DCS} }{11812864 \eta ^{14/5}}(-2 \text{$\delta_m$} \text{$\chi_a$} \text{$\chi_s$}+\left(1-\frac{16068 \eta }{61969}\right) \text{$\chi_a$}^2+\left(1-\frac{231808 \eta }{61969}\right) \text{$\chi_s$}^2)$ &\tiny -1 &\tiny $\bm{\frac{185627 \zeta_{DCS} }{1107456 \eta ^{14/5}}(-2 \text{$\delta_m$} \text{$\chi_a$} \text{$\chi_s$}+\left(1-\frac{53408 \eta }{14279}\right) \text{$\chi_a$}^2+\left(1-\frac{3708 \eta }{14279}\right) \text{$\chi_s$}^2)}$& \tiny 4\\
\hline
 \vspace{20pt}
\tiny KG&\rule{0pt}{4ex}\tiny$\bm{\frac{-5 \sqrt{\alpha^{KG}}\bigg(\frac{(\beta^{KG}-1)(2+\beta^{KG}+3\lambda^{KG})}{(\alpha^{KG}-2)(\beta^{KG}+\lambda^{KG})}\bigg)^{3/2}\eta ^{2/5} (\text{$s_1$}-\text{$s_2$})^2}{3584(\text{$G_N$} (\text{$s_1$}-1) (\text{$s_2$}-1))^{4/3}}}$&\tiny-7 &\tiny$\bm{\frac{112 }{3}\beta_{KG}}$&\tiny-2\\
\hline
 \vspace{20pt}
 \tiny NC&\rule{0pt}{4ex}\tiny${-\frac{75 (2 \eta -1) \Lambda ^2}{256 \eta ^{4/5}}}$&\tiny-1&\tiny$\bm{-\frac{3 (2 \eta -1) \Lambda ^2}{8 \eta ^{4/5}}}$&\tiny4\\
 \hline
  \vspace{20pt}
\tiny Einstein-$\AE$ther&\rule{0pt}{4ex}\tiny$\bm{-\frac{5 \eta ^{2/5} \left(s_1-s_2\right){}^2 \left(\left(c_{14}-2\right) w_0^3-w_1^3\right)}{3584 c_{14} w_0^3 w_1^3 \left(\text{$G_N$} \left(s_1-1\right) \left(s_2-1\right)\right){}^{4/3}}}$&\tiny-7&\tiny$\bm{-\frac{5 \eta ^{2/5} \left(s_1-s_2\right){}^2 \left(\left(c_{14}-2\right) w_0^3-w_1^3\right)}{96 c_{14} w_0^3 w_1^3 \left(\text{$G_N$} \left(s_1-1\right) \left(s_2-1\right)\right){}^{4/3}}}$&\tiny-2\\

 \hline
 \vspace{20pt}
 \tiny Varying-G Theory&\rule{0pt}{4ex}\tiny $\bm{-\frac{75 M_0 {\eta_0}^{3/5}}{851968 {G_0}^{10/3}} \bigg(\frac{11 \dot{G}}{3 G_0} + \frac{17 \dot{M}}{3M_0}-\frac{M_{1,0}\dot{M_2}+M_{2,0}\dot{M_1}}{{M_0}^2 \eta0}\bigg)}$&\tiny-13&\tiny$\bm{\frac{-5 M_0 {\eta_0}^{3/5}}{512 {G_0}^{5/3}} \bigg(\frac{7 \dot{G}}{ G_0} + \frac{5\dot{M}}{M_0}+\frac{M_{1,0}\dot{M_2}+M_{2,0}\dot{M_1}}{{M_0}^2 \eta0}\bigg)}$&\tiny-8\\
\hline
\end{tabular}

\include{reference}
\bibliographystyle{plain}
\bibliography{bibfile}
\end{document}